The abundance estimation problem described in Section \ref{Algorithm Unmixing} outputs a matrix $ \mathbf{A} \in \mathbb{R}_+^{n_e \times n_p}$ whose columns $\mathbf{a}_i$ hold the abundance values for the superpixel $\mathbf{c}_i$ with respect to the mean spectral signatures. The initial partitioning of the set of superpixels performed in Section \ref{Algorithm NCuts} often provides an adequate solution, however, there is always hope to further refine the segmentation (\cite{SSBD}). In the effort to do so, a new superpixel feature vector is created by concatenating the original spectral features and the abundance estimation results
\begin{equation}
    \label{Feature Vector Creation}
    \tilde{\mathbf{c}}_i = \mathbf{c}_i \oplus \mathbf{a}_i
\end{equation}

With the new construction of $\tilde{\mathbf{c}}$, the segmentation described in Section \ref{Algorithm NCuts} is repeated with the same set of parameters $\sigma$ and $\kappa$. The overall intuition behind this final step is to combine these additional, rich abundance features and the original spectral information to produce a more accurate segmentation. This proves useful along the boundaries between regions, where the creation of superpixels leads to mixed pixels. The added abundance information aids in adequately segmenting these mixed regions. The further refinement through the inclusion of the abundance information allows stronger spatial coherence in the clusters, due to the use of the graph regularization in \eqref{unmixing:graph-reg-ae}.

