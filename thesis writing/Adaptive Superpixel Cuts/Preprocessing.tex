As mentioned in Section \ref{Cube}, the input to the algorithm is a hyperspectral image represented by a nonnegative tensor $\mathbf{X}$  of shape $(n_x, n_y, n_\lambda)$. Raw hyperspectral images are susceptible to outliers and scale invariance across wavelengths. As such, to ensure algorithms perform reliably along a wide range of domains, preprocessing is a crucial step. Typically, the first main goal of practioners is to deal with noise across the spectral dimension, then work to deal with spatial artifacts in the image. This section will cover the Singular Value Decomposition and Layer Normalization methods that form the basis of the spectral preprocessing methods for the algorithm.

% $\mathbf{X}$ can be alternatively represented using matrix $\mathbf{X}_f \in \mathbb{R}_+^{(n_p, n_\lambda)}$, where $n_p = n_x n_y$ is the total number of pixels in the image, with each column representing the pixel at index $(i,j)$ in $\mathbf{X}$, arranged across the first spectral dimension, then the second. Formally, $
% \mathbf{X}_f = \left[ \mathbf{x}_{(0,0)},  \cdots, \mathbf{x}_{(n_x,0)}, \cdots, \mathbf{x}_{(0,n_y)}, \cdots, \mathbf{x}_{(n_x, n_y)} \right] $. 

