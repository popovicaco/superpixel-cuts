In Section \ref{Spectral Clustering}, the Normalized Cuts algorithm was introduced for the task of bipartitioning a group of pixels through creating an affinity matrix and solving the relaxed eigensystem in \eqref{sc:ncuts-formula}. Considering a matrix of superpixels $\mathbf{C}$ from the results of the SLIC algorithm in Section \ref{Algorithm Superpixels}, two matrices are formed. The first matrix is the spectral affinity matrix given by calculating the cosine similarity between the spectral features of the superpixels:
\begin{equation}
    \label{nc:spectral-mtx}
    \mathbf{W_{\text{spectral}}}_{(i,j)} = \arccos\left(\frac{\mathbf{c}_i \mathbf{c}_j^T}{\|\mathbf{c}_i\|_2\|\mathbf{c}_j\|_2}\right).
\end{equation}
The second matrix is the spatial distance matrix given by calculating the spatial euclidean distance between the superpixels:
\begin{equation}
    \label{nc:spatial-mtx}
    \mathbf{W_{\text{spatial}}}_{(i,j)} = d_{\text{spatial}}(\mathbf{c}_i, \mathbf{c}_j).
\end{equation}
To combine the spatial and spectral information within the image, a spectral similarity parameter $\sigma > 0$ and spatial limit parameter $\kappa > 0$ are introduced and the spatial-spectral affinity matrix $\mathbf{W}$ is then constrcuted as follows
\begin{equation}
    \label{nc:spatial-spectral-mtx}
    \mathbf{W}_{(i,j)} = \begin{cases}
        \exp\left(-\frac{\mathbf{W_{\text{spectral}}}_{(i,j)}^2}{\sigma^2}\right) &\quad \text{if } \mathbf{W_{\text{spatial}}}_{(i,j)} \leq \kappa\\
        0 &\quad \text{if } \mathbf{W_{\text{spatial}}}_{(i,j)} > \kappa.
    \end{cases}
\end{equation}
The intuiton behind constructing the spatial-spectral affinity matrix is to calculate spectral similarity between two superpixels if and only if the centroids of the superpixels are within a spatial range $\kappa$ of each other. This introduces spatial compactness within the partitioning.

After constructing the spatial-spectral affinity matrix, the goal is to to then utilize the normalized cuts algorithm to recursively bipartition the graph $G_\mathbf{W}$ represented by $\mathbf{W}$ into $n_e$ subgraphs. Doing so provides a segmentation of the columns of $\mathbf{C}$ representing the superpixels into $n_e$ clusters. Using the spatial-spectral affinity matrix $\mathbf{W}$, the diagonal matrix $\mathbf{D}$ is calculated according \eqref{sc:d-mtx} and the initial bipartitioning of $G_\mathbf{W}$ can be determined by solving for the second smallest eigenvalue $\lambda_2$ and the corresponding eigenvector $\mathbf{u}_2$ in the system given in \eqref{sc:ncuts-formula}. Bipartitioning the graph $G_\mathbf{W}$ according the sign of the entries in $\mathbf{u}_2$, the next partition is given by the one that minimizes \eqref{sc:ncut-criteria} within the two subgraphs. This process is continued until the graph $G_\mathbf{W}$ is partitioned into $n_e$ subgraphs. Cluster membership of the superpixels in $\mathbf{C}$ are assigned to the corresponding subgraph they belong to.

\begin{algorithm}[H]
    \caption{Spatial Spectral Clustering}
    \textbf{Input}: Superpixel Matrix $\mathbf{C}$, $\kappa > 0$, $\sigma > 0$, $n_e \geq 2$.

    \textbf{Initialize:} Construct the spatial-spectral affinity matrix $\mathbf{W}$ and diagonal matrix $\mathbf{D}$ according to \eqref{nc:spatial-spectral-mtx} and \eqref{sc:d-mtx}.\\

    \For{$k = 2$ \KwTo $n_e$}{ 
        For each subgraph, solve the system \eqref{sc:ncuts-formula}. Bipartition the subgraph according to the cut that minimizes \eqref{sc:ncut-criteria}. 
    }

    \textbf{Output}: Assign superpixel cluster memberships to a vector $\mathbf{v}_i \in \{0,1,\dots,n_e\}$ according to the subgraph each node belongs to.
\end{algorithm}

The algorithm allows for a flexible and efficient framework for segmentation tasks. The initial construction of the affinity matrix according to \eqref{nc:spatial-spectral-mtx} needs to only be done once, with subsequent subsegmentations being done using selected columns and rows corresponding the the subgraphs each node belongs to. The most obvious bottleneck in the algorithm is the step in which the eigensystem is solved, scalling cubically with the number of superpixels $n_s$. The lower the number of superpixels, the faster the algorithm performs. On the other end, the higher the number of superpixels, the slower the algorithm performs. The final result is determined by tuning $\sigma$ and $\kappa$. The lower $\sigma$ is, the more of an emphasis the spectral features have on the final result, while, the lower $\kappa$ is, the more of an emphasis the spatial information has on the final result. Careful selection of the two parameters allows for meaningful results.
