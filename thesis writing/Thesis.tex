\documentclass[10pt]{article}
\usepackage{graphicx} % Required for inserting images
\usepackage{amsmath}
\usepackage{amssymb}
\usepackage[margin=0.75in]{geometry}
\usepackage{algorithm2e}
\usepackage{fancyhdr}
\usepackage{caption}
\usepackage{float}

%############################ SETTINGS
\RestyleAlgo{ruled}
% \pagestyle{fancy} % Apply headers to all pages
% \fancyhf{} % Clear all header and footer definitions
% % \fancyhead[L]{\small Adaptive Superpixel Cuts for Hyperspectral Images}
% % \fancyhead[R]{\small \thepage}
\DeclareMathOperator*{\argmax}{arg\,max}
\DeclareMathOperator*{\argmin}{arg\,min \;\;}
\DeclareMathOperator*{\diag}{diag}
\DeclareMathOperator*{\proj}{\text{proj}}
\DeclareMathOperator*{\cut}{\text{cut}}
\DeclareMathOperator*{\ncut}{\text{ncut}}
\DeclareMathOperator*{\assoc}{\text{assoc}}
% #############################################
% 
% 
% 
% 
% 
% #############################################
\title{Adaptive Superpixel Cuts for Hyperspectral Images}
\author{Aleksandar Popovic}
\date{February 2024}

\begin{document}

\maketitle

\begin{abstract}
    Blind segmentation in hyperspectral images is a challenging problem. Many traditional methods suffer from poor identification of materials and expensive computational costs, which can be partially eased by trading the accuracy with efficiency.

    In this paper, we propose a novel graph-based algorithm for segmentation in hyperspectral images. Utilizing the fact that pixels in a local region are likely to have similar spectral features, a pre-clustering algorithm is used to extract the homogenous regions, called superpixels. After extracting the superpixels, a weighted graph is constructed with the weights representing both the spectral similarity and spatial distance between each superpixel and its neighbors. The normalized graph cuts algorithm is then used to perform an initial segmentation of the image. To effectively extract the material information in the superpixels, the mean spectra in each segment is used to estimate the abundance of each endmember in each superpixel using a graph regularized hyperspectral unmixing algorithm. The resulting abundance information is used as a supportive feature, which when combined with the spectral features, form a new spectral feature vector for each superpixel. Using this new feature vector, the weighted graph is once again constructed and the normalized cuts algorithm is applied, resulting in a final segmentation of the image.

    Experiments on a real hyperspectral datasets illustrate great potential of the proposed method in terms of accuracy and efficiency.
\end{abstract}

\clearpage
% #############################################
% 
% 
% 
% 
% 
% #############################################
\tableofcontents

\clearpage
% #############################################
% 
% 
% 
% 
% 
% #############################################
\section{Introduction}
This section focuses on introducing the proposed unsupervised hyperspectral image segmentation algorithm Adaptive Superpixel Cuts for Hyperspectral Images (ASC-HSI). This algorithm leverages the algorithms proposed in Section \ref{Background} to perform image segmentation on a superpixel-basis rather than a pixel-basis, reducing computational and memory costs while allowing for the use of a graph based approach to segmentation and unmixing. The steps of the algorithm can depicted as follows:
\begin{figure}[h]
    \centering % This centers the image
    \includegraphics[scale=0.4]{algorithm_view.png}  % Replace filename with your image file name (without extension)
    \label{fig:label}  % Optional label for referencing the figure in the text
  \end{figure}
  

% \subsection{Applications to Biomedical Imaging}
% \input{Introduction/BHSI.tex}

\clearpage
% #############################################
% 
% 
% 
% 
% 
% #############################################
\section{Background} \label{Background}
\input{Background/Background.tex}
% \subsection{Hyperspectral Imaging: Basics} \label{Basics}
% In hyperspectral imaging, near contiguous narrow band spectral information is measured for each spatial pixel of an image collected over a scene. Spectral information can be quantified by multiple measures. Traditionally, spectral radiance, being the energy emitted or reflected by a surface over a large number of spectral wavelengths, has been the measure of choice for physical applications. Utilizing this spectral information, physical and chemical properies of materials can be deduced and insights can be made about the overall composition of the images, resulting in applications across various fields.

Analysis in hyperspectral images has traditionally been a computationally expensive and difficult task due to algorithms scaling in both the spatial and spectral resolution of the images. This section will focus on building a relevant background for common preclustering, abundance estimation, and segmentation techniques that are common in hyperspectral image analysis.


\subsection{The Hyperspectral Cube} \label{Cube}
In traditional, RGB based imaging systems, an image can be represented by a 3-dimensional tensor of shape $(n_x, n_y, 3)$, where the last dimension corresponds to the color channel the image was captured in. From a mathematical point of view, a hyperspectral image, denoted by $\mathbf{X}$, is a tensor of shape $(n_x, n_y, n_\lambda)$ with nonnegative entries. Each pixel in the tensor is represented using a vector $\mathbf{x} \in \mathbb{R}_+^{n_\lambda}$. From a physical point of view, the first two dimensions in $\mathbf{X}$ represent the spatial coordinates  of the pixels, while the last dimension represents the specific wavelength band the spectral intensity, reflectance or transmittance measurements were taken at.


% \clearpage
\subsection{Superpixel Generation} \label{Superpixel}
In \ref{SLIC}, the SLIC algorithm was introduced, aiming to perceptually group pixels into locally homogeneous groups called superpixels. While the SLIC algorithm was originally developed for use in the CIELAB colorspace, a similar methodology can be applied towards the hyperspectral space. The main change when trasitioning to the hyperspectral domain requires the consideration of all the spectral features.

With the preprocessed hyperspectral image $\hat{\mathbf{X}}$, each pixel is given by the vector $\hat{\mathbf{x}}_{(i,j)} = [\hat{x}_1, \hat{x}_2, \dots, \hat{x}_{n_\lambda}]$. Then, taking as an input the number of superpixels $n_s$, centers $\mathbf{c}_n = [c_1, c_2, \dots, c_{n_\lambda}]$ where $n = 1, 2 \dots, n_s$ are created at regular grid intervals $S = \sqrt{n_s/n_p}$ across the image. The initial clusters are moved to the lowest gradient position in a $3 \times 3$ spatial neighborhood where the image gradient is now calculated using the original hyperspectral features instead of the CIELAB features: 
\begin{equation}
    \label{eq:slic-gradient-2}
    \mathbb{G}(i,j) = \|\hat{\mathbf{x}}_{(i+1,j)} - \hat{\mathbf{x}}_{(i-1,j)} \|_2^2 + \|\hat{\mathbf{x}}_{(i,j+1)} - \hat{\mathbf{x}}_{(i,j-1)} \|_2^2
\end{equation}

Following the original formulation of the SLIC algorithm, a modified distance measure is proposed to enforce color similarity and spatial extent within the superpixels. Using the same parameter $m$ to control the compactness and shape of the superpixels, the modified distance between a pixel $\hat{\mathbf{x}}$ and cluster $\mathbf{c}_n$ is now calculated as $L_2$ difference between the spectral features plus a scaled version of the spatial euclidean distance between the pixel and the cluster center:
\begin{equation}
    \label{eq:slic-cielab-distance-hsi}
    \mathbb{D}(\hat{\mathbf{x}}, \mathbf{c}_n) = \|\hat{\mathbf{x}} - \mathbf{c}_n\|_2^2 + \frac{m}{S}d_{\text{spatial}}(\hat{\mathbf{x}}, \mathbf{c}_n)^2
\end{equation}

Each pixel is associated with the nearest cluster $\mathbf{c}_n$ whose search area overlaps the pixel. After all pixels are associated with a cluster, a new cluster center is computed as the average vector of all the pixels belonging to the cluster. This is repeated for a set number of iterations. In the hyperspectral version of the SLIC algorithm, the option of relabelling disjoint segments is not performed, instead opting for higher selection of the $m$ and $n_s$ parameters to avoid disjoint segments all together. Once the algorithm is completed, the final superpixeled image is given by arranging the feature vectors into columns of the matrix $\mathbf{C} = [\mathbf{c}_1 | \mathbf{c}_2 | \dots | \mathbf{c}_{n_s}]$. 

\begin{algorithm}[H]
    \caption{Hyperspectral SLIC Algorithm}
    \textbf{Input}: $m > 0$, $n_s > 0$, $n_{\text{iters}} > 0$, Preprocessed Hyperspectral Image $\hat{\mathbf{X}}$.

    \textbf{Initialize:} $\mathbf{c}_n = [c_{l}, c_{a}, c_{b}]$ where $n = 1, \cdots, n_s$ by sampling pixels at regular grid intervals $S$. Perturb cluster centers to lowest gradient position in a $3 \times 3$ neighborhood according to \eqref{eq:slic-gradient-2}. \\
    
    \For{$k = 1$ \KwTo $n_{\text{iters}}$}{ 
        Assign best matching pixels from a $2S \times 2S$ neighborhood around clusters $\mathbf{c}_k$ according to \eqref{eq:slic-cielab-distance-hsi}.\\
         Compute new cluster centers according to average of all pixels belonging to cluster.
    }
    \textbf{Output}: Superpixeled Image Matrix $\mathbf{C}$
\end{algorithm}


In the hyperspectral domain, superpixels are slightly less adept at creating visually meaningful partitionings due to use of the raw hyperspectral spectra rather than a perceptual colorspace like CIELAB. To alleviate this, higher values of $m$ and $n_s$ are used to form more spatially compact regions akin to the ones formed in the original algorithm. Nonetheless, superpixels are valuable in allowing practioners to avoid having to consider the variation between individual pixels and instead consider the variation between these new perceptual groupings of pixels. Superpixels are exceptionally useful for dealing with artifacts or imaging deficits in simple instances.

SHOW RESULTS OF SUPERPIXELING RESULTS FOR DIFFERENT VALUES OF NS AND NP
\subsubsection{Simple Linear Iterative Clustering} \label{SLIC}
In this section, we will introduce the Simple Linear Iterative Clustering (SLIC) algorithm. The algorithm is a special case of the k-means algorithm adapted to the task generating superpixels in a $5$-dimensional space, where the first $3$ dimensions correspond to the the pixel color vector in the CIELAB colorpsace, and last $2$ dimensions correspond to the spatial coordinates $(i,j)$ of the pixel in the image.. Formally, each pixel $\mathbf{x}_{(i,j)} = [x_l, x_a, x_b]$ is restructured into the form $\tilde{\mathbf{x}}_{(i,j)} = [x_l, x_a, x_b, i, j]$. With this modified feature vector $\tilde{\mathbf{x}}$, we incorporate both spectral and spatial information into the clustering, however, while the spectral information has bounds on it's values, the spatial information depends on the size of the image. 

Taking as an input the desired number of superpixels $n_s$, for an image with $n_p = n_x n_y$ pixels, each superpixel would be composed of approximately $n_s / n_p$ pixels. Assuming the superpixels lie on a grid, a superpixel centroid would occur at every grid interval $S = \sqrt{n_s/n_p}$. At the onset of the algorithm, a grid of $n_s$ superpixel centers $\mathbf{C}_n = [c_{l}, c_{a}, c_{b}, i, j]$ where $n = 1, \cdots, n_s$ are sampled across the image with regular grid intervals $S$. To avoid sampling noisy pixels, clusters are moved to the lowest gradient position in a $3 \times 3$ neighborhood where the image gradient is calcuated, using the original spectral vector $x$ in the CIELAB color space as:
\begin{equation}
    \label{eq:slic-gradient}
    \mathbb{G}(i,j) = \|\mathbf{x}_{(i+1,j)} - \mathbf{x}_{(i-1,j)} \|^2 + \|\mathbf{x}_{(i,j+1)} - \mathbf{x}_{(i,j-1)} \|^2
\end{equation}

After initialization, a modified distance measure is proposed to enforce color similarity and spatial extent within the superpixels. Since the approximate area of each superpixel is $S^2$, it is assumed that pixels associated with a superpixel lie within a $2S \times 2S$ neighborhood of the superpixel centroid. Introducing the parameter $m$ to control the compactness and shape of the superpixels, the modified distance is then calculated as 
\begin{equation}
    \label{eq:slic-cielab-distance}
    \mathbb{D}(x, y) = \|\mathbf{x}_{lab} - \mathbf{y}_{lab}\|^2 + \frac{m}{S}\|\mathbf{x}_{ij} - \mathbf{y}_{ij}\|^2
\end{equation}

Each pixel is the image is associated with the nearest cluster whose search area overlaps this pixel. After all pixels are associated with a cluster, a new center is computed as the average feature vector of all the pixels belonging to the cluster. This is repeated for a set number of iterations. After exhausting all iterations, a final step is performed by relabelling disjoint segments with the labels of the largest neighboring cluster. This step is optional as disjoint segments tend to not occur for larger inputs of $n_s$ and $m$.

\begin{algorithm}[H]
    \caption{SLIC Superpixel Algorithm}
    \textbf{Input}: $m > 0$, $n_s > 0$, $n_{\text{iters}} > 0$, CIELAB Image $\mathbf{X}$.

    \textbf{Initialize:} $\mathbf{c}_n = [c_{l}, c_{a}, c_{b}, i, j]$ where $n = 1, \cdots, n_s$ by sampling pixels at regular grid intervals $S$. Perturb cluster centers to lowest gradient position in a $3 \times 3$ neighborhood according to \eqref{eq:slic-gradient} \\
    
    \For{$k = 1$ \KwTo $n_{\text{iters}}$}{ 
        Assign best matching pixels from a $2S \times 2S$ neighborhood around clusters $C_k$ according to distance measure \eqref{eq:slic-cielab-distance}. \\
        Compute new cluster centers according to average vector of all pixels belonging to cluster.
    }
    \textbf{Optional:} Relabel disjoint segments.\\

    \textbf{Output}: Superpixeled Image $\mathbf{X}_s = [\mathbf{c}_1, \mathbf{c}_2, \dots, \mathbf{c}_{n_s}]$
\end{algorithm}

The SLIC algorithm is shown to produce meaningful and noise-robust segments in traditional computer vision applications. This algorithm proves useful in Section \ref{Algorithm Superpixels} when adapted as a pre-clustering step in the hyperspectral domain.


% \clearpage
\subsection{Hyperspectral Unmixing} \label{Unmixing Intro}
In hyperspectral imaging applications, there is emphasis on estimating the relative abundance of a given representantive material, called an endmember, within each pixel.  Unmixing results often give more detailed information about the overall composition of a hyperspectral scene with set of selected endmembers. This section will introduce the foundational knowledge behind unmixing and abundance estimation. 


\subsubsection{Linear Mixing Model}\label{LMM}
\input{Background/Hyperspectral Unmixing/Linear Mixing Model.tex}
\subsubsection{Abundance Estimation}\label{AE}
Given the linear mixing model (\eqref{lmm:abund-lmm-collection}), in traditional hyperspectral imaging tasks, both $\mathbf{M}$ and $\mathbf{A}$ are unknown. Often, researchers aim to estimate $\mathbf{M}$ first, as spectral signatures collected from endmembers in same scene under the same conditions will be almost identical. Notably, in the field of remote sensing, effort has been made to create a library of spectral signatures derived from common vegetation and minerals in land cover images, allowing focus to be made solely in estimating $\mathbf{A}$ (\cite{ECOSTRESS}). This section will cover the scenario where $\mathbf{M}$ is known and $\mathbf{A}$ is to be estimated. The task is referred to as abundance estimation and continues to be an active area of research, where the aim is to find $\mathbf{A}$ such that an error function $\mathcal{L}$ is minimized with respect to the reconstructed collections of pixels $\tilde{\mathbf{X}} = \mathbf{MA}$ and the original collection of pixels $\mathbf{X}$. Traditionally, we aim to minimize the least-square reconstruction error between the entries in $\tilde{\mathbf{X}}$ and $\mathbf{X}$

\begin{equation}
    \label{ae:fnorm}
    \mathcal{L}(\mathbf{X},\tilde{\mathbf{X}}) = \sum_{i=1}^{n_\lambda} \sum_{j=1}^{n_p} \left(\mathbf{x}_{(i,j)} - \tilde{\mathbf{x}}_{(i,j)}\right)^2 = \|\tilde{\mathbf{X}} - \mathbf{X}\|_F^2 
\end{equation}

The least-squares reconstruction error can alternatively be written as the squared Frobenius norm, denoted as $\|\cdot\|_F^2$, of the difference between  $\tilde{\mathbf{X}}$ and $\mathbf{X}$. This choice of $\mathcal{L}$ is the straight forward and natural approach as $\mathcal{L}$ is both convex and differentiable, with the additional properties that $\mathcal{L}(\tilde{\mathbf{X}},\mathbf{X}) = \mathcal{L}(\mathbf{X},\tilde{\mathbf{X}})$ and $ \mathcal{L}(\mathbf{X},\tilde{\mathbf{X}}) = \mathcal{L}(\mathbf{X}^T,\tilde{\mathbf{X}^T})$ (\cite{UNMIX}).

To incorporate the ANC-ASC constraint into the overall formulation of $\mathcal{L}$, the set $\Delta$ from Section \ref{LMM} plays a significant role. It is important to note that $\Delta$ is a convex set, meaning that for matrices $A, B \in \Delta$, for all $0 \leq \alpha \leq 1$, the matrix $C = \alpha A + (1- \alpha)B$ is also an element of $\Delta$. The inclusion of the constraints on $\mathbf{A}$ is facilitated using the piece wise function $\chi_S$ defined as follows

\begin{equation}
    \label{ae:indfunc}
    \chi_{S}(x) = 
            \begin{cases}
            0 &\text{if } x \in S \\
            \infty &\text{if } x \not \in S 
            \end{cases}
\end{equation}

Adding $\chi_\Delta$ in the formulation of $\mathcal{L}$ restricts the values $\mathbf{A}$ can take on to the set $\Delta$, while ensuring that the overall formulation still has a global minimum within $\Delta$. Additionally, a regularization term $J$ can be added to impose additional constraints on the values in $\mathbf{A}$. Formally, abundance estimation can be formulated as a convex optimization problem of the form 
\begin{equation}
    \label{ae:ae-min-1}
    \widehat{\mathbf{A}} = \argmin_{\mathbf{A} \in \mathbb{R}^{n_e \times n_p}} \frac{1}{2}\|\mathbf{MA} - \mathbf{X}\|_F^2 + \chi_\Delta(\mathbf{A}) + J(\mathbf{A}) 
\end{equation}

This problem has no closed form solution, relying on iterative methods or applying solvers to solve the problem for individual pixels given the problem can be split pixel wise. The formulation of the abundance estimation problems allows for additional image specifications to be added depending on the domain of research. 

% \clearpage
\subsection{Alternating Direction Method of Multipliers}\label{ADMM Intro}
\input{Background/ADMM/ADMM.tex}
% \subsubsection{Abundance Estimation using ADMM}\label{Block ADMM}
%  In Section \ref{AE}, the abundance estimation problem for a collection of pixels $\mathbf{X}$ given the endmember spectra matrix $\mathbf{M}$ was known was stated as follows:

\begin{equation*}
    A = \argmin_{A \in \mathbb{R}^{n_e \times n_p}} \frac{1}{2}\|\mathbf{MA} - \mathbf{X}\|_F^2 + \chi_\Delta(\mathbf{A}) + J(\mathbf{A}).
\end{equation*}

The goal of this section is to demonstrate how this problem can be equivalently represented in a form where ADMM can directly be applied. The section itself operates under the assumption that $J$ is a convex function, for nonconvex choices of $J$, convergence is not guarenteed. The approach to transforming \eqref{ae:ae-min-1} is to first introduce matrices $\mathbf{U} \in \mathbb{R}^{n_e \times n_p}$, $\mathbf{V}_1 \in \mathbb{R}^{n_b \times n_p}$, $\mathbf{V}_2 \in \mathbb{R}^{n_e \times n_p}$, and $\mathbf{V}_{3} \in \mathbb{R}^{n_e \times n_p}$ and rewrite the problem into the equivalent form 
\begin{equation}
    \label{ae:equivalent-admm-1}
    \begin{aligned}
        \underset{\mathbf{U}, \mathbf{V}_1, \mathbf{V}_2, \mathbf{V}_2}{\textbf{minimize }} & \frac{1}{2} \|\mathbf{V}_1 - X \|_F^2 + \chi_{\Delta}(\mathbf{V}_2) + J(\mathbf{V}_3) 
        \\         
        \textbf{subject to } & \mathbf{V}_1 = \mathbf{MU} \\
        &\mathbf{V}_2 = \mathbf{U} \\
        &\mathbf{V}_{3} = \mathbf{U}.
   \end{aligned}
\end{equation}
Further rewriting the problem, the application of ADMM can clearly be seen:
\begin{equation}
    \label{ae:equivalent-admm-2}
    \begin{aligned}
        \underset{\mathbf{U},\mathbf{V}}{\textbf{minimize }} & g(\mathbf{V})
        \\         
        \textbf{subject to } & \mathbf{GU} + \mathbf{BV} = \mathbf{0}
   \end{aligned}
\end{equation}
where 
$
g(\mathbf{V}) = \frac{1}{2} \|\mathbf{V}_1 - X \|_F^2 + \chi_{\Delta}(\mathbf{V}_2) + J(\mathbf{V}_3) 
$, 
$$
\mathbf{V} = \begin{bmatrix}
\mathbf{V}_1 &  &   \\
 &\mathbf{V}_2&   \\
  &  & \mathbf{V}_3 \\
\end{bmatrix},
\quad
\mathbf{G} = 
\begin{bmatrix}
\mathbf{M}\\ 
\mathbf{I}\\ 
\mathbf{I}\\ 
\end{bmatrix},
\quad
\mathbf{B} = 
\begin{bmatrix}
-\mathbf{I} &  &  \\
  &-\mathbf{I}&  \\
&  & -\mathbf{I} \\
\end{bmatrix}.
$$

With parameter $\mu > 0$, the augmented lagrangian $\mathcal{L}_\mu$ for the abundance estimation problem 


% \clearpage
\subsection{Spectral Clustering} \label{Spectral Clustering}
\input{Background/Spectral Clustering/Spectral Clustering.tex}
\subsubsection{Normalized Cuts} \label{Normalized Cuts}
A graph $G = (V,E)$ with affinity matrix $\mathbf{W}$ can be partitioned into two subgraphs $G_A = (V_A, E_A)$ and $G_B = (V_B, E_B)$ such that $V_A \cup V_B = V$ and $V_A \cap V_B = \emptyset$ by removing the edges between the vertices in $G_A$ and $G_B$. The dissimilarity between these two graphs can be calculated as the sum of the edges cut to form the partitions
\begin{equation}
    \label{sc:cut}
    \cut(G_A,G_B) = \sum_{i \in V_A,\;j \in V_B}{\mathbf{W}_{(i,j)}}
\end{equation}
The optimal bi-partitioning of $G_W$ is given as the graphs $G_A$ and $G_B$ that minimize \eqref{sc:cut}. However, in the case of image segmentation, this criteria will heavily prioritize partitioning single pixels from the image. Instead, the normalized cuts criteria is proposed, focusing on balancing the ratio between the edges cut and the sum of the internal edge nodes within $G_A$ and $G_B$ defined as 
\begin{equation}
    \label{sc:ncut-criteria}
    \ncut(G_A, G_B) = \frac{\cut(G_A, G_B)}{\assoc(G_A, G)} + \frac{\cut(G_A, G_B)}{\assoc(G_B, G)}
\end{equation}
where 
\begin{equation}
    \label{sc:assoc}
    \assoc(G_A, G) = \sum_{i \in V_A,\;j \in V}{\mathbf{W}_{(i,j)}} 
\end{equation}
The normalized cuts criteria, in general terms, aims to minimize the disassociation between the subgraphs and maximize the association within them. While \eqref{sc:ncut-criteria} is NP-Complete, (\cite{NCUTS}) show that solving for the eigenvector $\mathbf{u}_2$ corresponding to the second smallest eigenvalue $\lambda_2$ in the system
\begin{equation}
    \label{sc:ncuts-formula}
    \mathbf{D}^{-\frac{1}{2}}(\mathbf{D} - \mathbf{W})\mathbf{D}^{-\frac{1}{2}}\mathbf{z} = \lambda \mathbf{z}
\end{equation}
provides a approximate real valued solution to \eqref{sc:ncut-criteria} through assigning subgraph membership of the vertices according to the sign of the entries in $\mathbf{u}_2$.

Spectral graph techniques like the normalized cuts algorithm provide the advantage of flexible and deterministic results based on the initialization of the adjacency matrix $\mathbf{W}$, however begin to fall behind iterative methods when the graph is large, due to the time complexity of solving \eqref{sc:ncuts-formula} scaling cubically with the number of vertices. 

\clearpage
% #############################################
% 
% 
% 
% 
% 
% #############################################
\section{Adaptive Superpixel Cuts} \label{Algorithm Intro}
This section focuses on introducing the proposed unsupervised hyperspectral image segmentation algorithm Adaptive Superpixel Cuts for Hyperspectral Images (ASC-HSI). This algorithm leverages the algorithms proposed in Section \ref{Background} to perform image segmentation on a superpixel-basis rather than a pixel-basis, reducing computational and memory costs while allowing for the use of a graph based approach to segmentation and unmixing. The steps of the algorithm can depicted as follows:
\begin{figure}[h]
    \centering % This centers the image
    \includegraphics[scale=0.4]{algorithm_view.png}  % Replace filename with your image file name (without extension)
    \label{fig:label}  % Optional label for referencing the figure in the text
  \end{figure}
  


% \subsection{Proposed Algorithm} \label{Algorithm Overview}
\subsection{Dataset Preprocessing} \label{Algorithm Preprocessing}
As mentioned in Section \ref{Cube}, the input to the algorithm is a hyperspectral image represented by a nonnegative tensor $\hat{\mathbf{X}}$  of shape $(n_x, n_y, n_\lambda)$. Raw hyperspectral images are susceptible to outliers and scale invariance across wavelengths, as such, to ensure algorithms perform reliably along a wide range of domains, preprocessing is a crucial step. Typically, the first main goal of practioners is to deal with noise across the spectral dimension, then work to deal with spatial artifacts in the image. This section will cover the Singular Value Decomposition and Layer Normalization methods that form the basis of the spectral preprocessing methods for the algorithm.

% $\mathbf{X}$ can be alternatively represented using matrix $\mathbf{X}_f \in \mathbb{R}_+^{(n_p, n_\lambda)}$, where $n_p = n_x n_y$ is the total number of pixels in the image, with each column representing the pixel at index $(i,j)$ in $\mathbf{X}$, arranged across the first spectral dimension, then the second. Formally, $
% \mathbf{X}_f = \left[ \mathbf{x}_{(0,0)},  \cdots, \mathbf{x}_{(n_x,0)}, \cdots, \mathbf{x}_{(0,n_y)}, \cdots, \mathbf{x}_{(n_x, n_y)} \right] $. 




% The follow algorithm is proposed

\begin{figure}[h]
    \centering % This centers the image
    \includegraphics[scale=0.33]{algorithm_view.png}  % Replace filename with your image file name (without extension)
    \label{fig:label}  % Optional label for referencing the figure in the text
  \end{figure}
  
\subsection{Hyperspectral Superpixel Generation} \label{Algorithm Superpixels}
In \ref{SLIC}, the SLIC algorithm was introduced, aiming to perceptually group pixels into locally homogeneous groups called superpixels. While the SLIC algorithm was originally developed for use in the CIELAB colorspace, a similar methodology can be applied towards the hyperspectral space. The main change when trasitioning to the hyperspectral domain requires the consideration of all the spectral features.

With the preprocessed hyperspectral image $\hat{\mathbf{X}}$, each pixel is given by the vector $\hat{\mathbf{x}}_{(i,j)} = [\hat{x}_1, \hat{x}_2, \dots, \hat{x}_{n_\lambda}]$. Then, taking as an input the number of superpixels $n_s$, centers $\mathbf{c}_n = [c_1, c_2, \dots, c_{n_\lambda}]$ where $n = 1, 2 \dots, n_s$ are created at regular grid intervals $S = \sqrt{n_s/n_p}$ across the image. The initial clusters are moved to the lowest gradient position in a $3 \times 3$ spatial neighborhood where the image gradient is now calculated using the original hyperspectral features instead of the CIELAB features: 
\begin{equation}
    \label{eq:slic-gradient-2}
    \mathbb{G}(i,j) = \|\hat{\mathbf{x}}_{(i+1,j)} - \hat{\mathbf{x}}_{(i-1,j)} \|_2^2 + \|\hat{\mathbf{x}}_{(i,j+1)} - \hat{\mathbf{x}}_{(i,j-1)} \|_2^2
\end{equation}

Following the original formulation of the SLIC algorithm, a modified distance measure is proposed to enforce color similarity and spatial extent within the superpixels. Using the same parameter $m$ to control the compactness and shape of the superpixels, the modified distance between a pixel $\hat{\mathbf{x}}$ and cluster $\mathbf{c}_n$ is now calculated as $L_2$ difference between the spectral features plus a scaled version of the spatial euclidean distance between the pixel and the cluster center:
\begin{equation}
    \label{eq:slic-cielab-distance-hsi}
    \mathbb{D}(\hat{\mathbf{x}}, \mathbf{c}_n) = \|\hat{\mathbf{x}} - \mathbf{c}_n\|_2^2 + \frac{m}{S}d_{\text{spatial}}(\hat{\mathbf{x}}, \mathbf{c}_n)^2
\end{equation}

Each pixel is associated with the nearest cluster $\mathbf{c}_n$ whose search area overlaps the pixel. After all pixels are associated with a cluster, a new cluster center is computed as the average vector of all the pixels belonging to the cluster. This is repeated for a set number of iterations. In the hyperspectral version of the SLIC algorithm, the option of relabelling disjoint segments is not performed, instead opting for higher selection of the $m$ and $n_s$ parameters to avoid disjoint segments all together. Once the algorithm is completed, the final superpixeled image is given by arranging the feature vectors into columns of the matrix $\mathbf{C} = [\mathbf{c}_1 | \mathbf{c}_2 | \dots | \mathbf{c}_{n_s}]$. 

\begin{algorithm}[H]
    \caption{Hyperspectral SLIC Algorithm}
    \textbf{Input}: $m > 0$, $n_s > 0$, $n_{\text{iters}} > 0$, Preprocessed Hyperspectral Image $\hat{\mathbf{X}}$.

    \textbf{Initialize:} $\mathbf{c}_n = [c_{l}, c_{a}, c_{b}]$ where $n = 1, \cdots, n_s$ by sampling pixels at regular grid intervals $S$. Perturb cluster centers to lowest gradient position in a $3 \times 3$ neighborhood according to \eqref{eq:slic-gradient-2}. \\
    
    \For{$k = 1$ \KwTo $n_{\text{iters}}$}{ 
        Assign best matching pixels from a $2S \times 2S$ neighborhood around clusters $\mathbf{c}_k$ according to \eqref{eq:slic-cielab-distance-hsi}.\\
         Compute new cluster centers according to average of all pixels belonging to cluster.
    }
    \textbf{Output}: Superpixeled Image Matrix $\mathbf{C}$
\end{algorithm}


In the hyperspectral domain, superpixels are slightly less adept at creating visually meaningful partitionings due to use of the raw hyperspectral spectra rather than a perceptual colorspace like CIELAB. To alleviate this, higher values of $m$ and $n_s$ are used to form more spatially compact regions akin to the ones formed in the original algorithm. Nonetheless, superpixels are valuable in allowing practioners to avoid having to consider the variation between individual pixels and instead consider the variation between these new perceptual groupings of pixels. Superpixels are exceptionally useful for dealing with artifacts or imaging deficits in simple instances.

SHOW RESULTS OF SUPERPIXELING RESULTS FOR DIFFERENT VALUES OF NS AND NP


\subsection{Spatial Spectral Clustering}\label{Algorithm NCuts}
In Section \ref{Spectral Clustering}, the Normalized Cuts algorithm was introduced for the task of bipartitioning a group of pixels through creating an affinity matrix and solving the relaxed eigensystem in \eqref{sc:ncuts-formula}. Considering a matrix of superpixels $\mathbf{C}$ from the results of the SLIC algorithm in Section \ref{Algorithm Superpixels}, two matrices are formed. The first matrix is the spectral affinity matrix $\mathbf{W_{\text{spectral}}}$ given by calculating the spectral angle between the spectral features of each pair of superpixels:
\begin{equation}
    \label{nc:spectral-mtx}
    \mathbf{W_{\text{spectral}}}_{(i,j)} = \arccos\left(\frac{\mathbf{c}_i \mathbf{c}_j^T}{\|\mathbf{c}_i\|_2\|\mathbf{c}_j\|_2}\right) 
\end{equation}
The second matrix is the spatial distance matrix $\mathbf{W_{\text{spatial}}}$ given by calculating the spatial euclidean distance between each pair of superpixels:
\begin{equation}
    \label{nc:spatial-mtx}
    \mathbf{W_{\text{spatial}}}_{(i,j)} = d_{\text{spatial}}(\mathbf{c}_i, \mathbf{c}_j) 
\end{equation}
To combine the spatial and spectral information within the image, a spectral similarity parameter $\sigma > 0$ and spatial limit parameter $\kappa > 0$ are introduced and the spatial-spectral affinity matrix $\mathbf{W}$ is then constructed as follows
\begin{equation}
    \label{nc:spatial-spectral-mtx}
    \mathbf{W}_{(i,j)} = \begin{cases}
        \exp\left(-\frac{\mathbf{W_{\text{spectral}}}_{(i,j)}^2}{\sigma^2}\right) &\quad \text{if } \mathbf{W_{\text{spatial}}}_{(i,j)} \leq \kappa\\
        0 &\quad \text{if } \mathbf{W_{\text{spatial}}}_{(i,j)} > \kappa 
    \end{cases}
\end{equation}
The intuition behind constructing the spatial-spectral affinity matrix is to calculate spectral similarity between two superpixels if and only if the centroids of the superpixels are within a spatial range $\kappa$ of each other. This introduces spatial compactness within the partitioning.

After constructing the spatial-spectral affinity matrix, the goal is to to then utilize the normalized cuts algorithm to recursively bi-partition the graph $G_\mathbf{W}$ represented by $\mathbf{W}$ into $n_e$ subgraphs. Doing so provides a segmentation of the columns of $\mathbf{C}$ representing the superpixels into $n_e$ clusters. Using the spatial-spectral affinity matrix $\mathbf{W}$, the diagonal matrix $\mathbf{D}$ is calculated according \eqref{sc:d-mtx} and the initial bi-partitioning of $G_\mathbf{W}$ can be determined by solving for the second smallest eigenvalue $\lambda_2$ and the corresponding eigenvector $\mathbf{u}_2$ in the system given in \eqref{sc:ncuts-formula}. Bi-partitioning the graph $G_\mathbf{W}$ according the sign of the entries in $\mathbf{u}_2$, the next partition is given by the one that minimizes \eqref{sc:ncut-criteria} within the two subgraphs. This process is continued until the graph $G_\mathbf{W}$ is partitioned into $n_e$ subgraphs. Cluster membership of the superpixels in $\mathbf{C}$ are assigned to the corresponding subgraph they belong to. Additionally, the mean spectral signatures of the superpixels within each cluster are calculated and arranged into the matrix $\mathbf{M}$.

\begin{algorithm}[H]
    \label{Spatial Spectral Segmentation}
    \caption{Spatial Spectral Segmentation}
    \textbf{Input}: Superpixel Matrix $\mathbf{C}$, $\kappa > 0$, $\sigma > 0$, $n_e \geq 2$.

    \textbf{Initialize:} Construct the spatial-spectral affinity matrix $\mathbf{W}$ and diagonal matrix $\mathbf{D}$ according to \eqref{nc:spatial-spectral-mtx} and \eqref{sc:d-mtx}.\\

    \textbf{Recursion:}\\
        \quad For each subgraph, solve the system \eqref{sc:ncuts-formula}. Bipartition the subgraph by assigning parition membership according to the cut that minimizes \eqref{sc:ncut-criteria}. 
    \\

    \textbf{Output}: Assign superpixel cluster memberships to a vector $\mathbf{v}_i \in \{1, 2, \dots ,n_e\}$ according to the subgraph each node belongs to. Form a endmember spectra matrix $\mathbf{M} = [ \mathbf{m}_1 | \mathbf{m}_2, | \dots | \mathbf{m}_{n_e} ]$ where $\mathbf{m}_i$ is the average spectral feature vector for all superpixels within the cluster $i$.
\end{algorithm}

The algorithm allows for a flexible and efficient framework for segmentation tasks. The initial construction of the affinity matrix according to \eqref{nc:spatial-spectral-mtx} needs to only be done once, with subsequent subsegmentations being done using selected columns and rows corresponding the the subgraphs each node belongs to. The most obvious bottleneck in the algorithm is the step in which the eigensystem is solved, scalling cubically with the number of superpixels $n_s$. The lower the number of superpixels, the faster the algorithm performs. On the other end, the higher the number of superpixels, the slower the algorithm performs. The final result is determined by tuning $\sigma$ and $\kappa$. The lower $\sigma$ is, the more of an emphasis the spectral features have on the final result, while, the lower $\kappa$ is, the more of an emphasis the spatial information has on the final result. Careful selection of the two parameters allows for meaningful results.


\subsection{Graph Regularized Abundance Estimation}\label{Algorithm Unmixing}
In Section \ref{AE}, the abundance estimation problem for a collection of pixels $\mathbf{X}$ given the endmember spectra matrix $\mathbf{M}$ was known was stated as follows:
\begin{equation*}
    \mathbf{A} = \argmin_{\mathbf{A} \in \mathbb{R}^{n_e \times n_s}} \frac{1}{2}\|\mathbf{MA} - \mathbf{X}\|_F^2 + \chi_\Delta(\mathbf{A}) + J(\mathbf{A}) 
\end{equation*}

The goal of this section is to apply a similar framework to the collection of superpixels $\mathbf{C}$ and determine estimates on the fractional abundances given an cluster spectra matrix $\mathbf{M}$ from the output of the clustering in Section \ref{Algorithm NCuts}. The abundance estimation problem in terms of superpixels can now be restated as
\begin{equation*}
    \mathbf{A} = \argmin_{\mathbf{A} \in \mathbb{R}^{n_e \times n_s}} \frac{1}{2}\|\mathbf{MA} - \mathbf{C}\|_F^2 + \chi_\Delta(\mathbf{A}) + J(\mathbf{A}) 
\end{equation*}

In previous sections, a regularization term $J$ was introduced to provide further control on the final values of $\mathbf{A}$. In imaging applications, a common assumption is that color values should typically not vary greatly for pixels next to each other. In a similar fashion, abundance values should not vary greatly for superpixels spatially close to each other (\cite{GraphL}). To accommodate this assumption, the matrix $\mathbf{W}_{\text{spatial}}$ given in \eqref{nc:spatial-mtx} can be exploited by considering
\begin{equation}
    \label{nc:spatial_filter_mtx}
    \mathbf{W}_{{\kappa}_{(i,j)}} = \begin{cases}
        1 &\quad \text{if } \mathbf{W_{\text{spatial}}}_{(i,j)} \leq \kappa\\
        0 &\quad \text{if } \mathbf{W_{\text{spatial}}}_{(i,j)} > \kappa 
    \end{cases}
\end{equation}
The regularization term
\begin{equation*}
    J(\mathbf{A}) = \frac{1}{2}\sum_{i = 1}^{n_s} \sum_{j = 1}^{n_s} \mathbf{W}_{{\kappa}_{(i,j)}} \|\mathbf{a}_i - \mathbf{a}_j\|_2^2
\end{equation*}
is minimized under the assumption that $\mathbf{a}_i$ and $\mathbf{a}_j$ should be similar if $\mathbf{c}_i$ and $\mathbf{c}_j$ are spatially within a distance $\kappa$ of each other. Importantly, $J$ is convex and differentiable and can alternatively be represented using the corresponding Laplacian matrix $\mathbf{L}$ described in \eqref{sc:laplacian-mtx} for the matrix $\mathbf{W}_{\kappa}$:
\begin{equation}
    \label{unmixing:distance-regularization}
    J(\mathbf{A}) = \frac{1}{2}\sum_{i = 1}^{n_s} \sum_{j = 1}^{n_s} \mathbf{W}_{{\kappa}_{(i,j)}} \|\mathbf{a}_i - \mathbf{a}_j\|_2^2 = \text{tr}(\mathbf{ALA}^T) 
\end{equation}

The graph regularized abundance estimation problem with the known cluster spectra matrix $\mathbf{M}$ and regularization weight parameter $\beta > 0$ is represented as:
\begin{equation}
    \label{unmixing:graph-reg-ae}
    \mathbf{A} = \argmin_{\mathbf{A} \in \mathbb{R}^{n_e \times n_s}} \frac{1}{2}\|\mathbf{MA} - \mathbf{C}\|_F^2 + \chi_\Delta(\mathbf{A}) + \frac{\beta}{2}\text{tr}(\mathbf{ALA}^T) 
\end{equation}
The goal of this rest of the section is to demonstrate how this problem can be equivalently represented in a form where the Alternating Direction Method of Multipliers technique can be applied. The abundance estimation problem when one or more than one regularization terms are added belongs to a class of problems called global consensus optimization problems shown in (\cite{ADMM}). The standard approach to transforming \eqref{unmixing:graph-reg-ae} is to introduce matrices $\mathbf{U} \in \mathbb{R}^{n_e \times n_s}$, $\mathbf{V}_1 \in \mathbb{R}^{n_b \times n_s}$, $\mathbf{V}_2 \in \mathbb{R}^{n_e \times n_s}$, and $\mathbf{V}_{3} \in \mathbb{R}^{n_e \times n_s}$ and rewrite the problem as:
\begin{equation}
    \label{unmixing:graph-reg-ae-admm-1}
    \begin{aligned}
        \underset{\mathbf{U}, \mathbf{V}_1, \mathbf{V}_2, \mathbf{V}_3}{\text{minimize }} & \quad \frac{1}{2} \|\mathbf{V}_1 - \mathbf{C} \|_F^2 + \chi_{\Delta}(\mathbf{V}_2) + \frac{\beta}{2}\text{tr}(\mathbf{V}_3 \mathbf{L} \mathbf{V}_3^T) 
        \\         
        \text{subject to } &  \quad \mathbf{V}_1 = \mathbf{MU} \\
        & \quad \mathbf{V}_2 = \mathbf{U} \\
        & \quad \mathbf{V}_{3} = \mathbf{U}
   \end{aligned}
\end{equation}

Further manipulation shows that by letting
$
g(\mathbf{V}) = \frac{1}{2} \|\mathbf{V}_1 - \mathbf{C} \|_F^2 + \chi_{\Delta}(\mathbf{V}_2) + \frac{\beta}{2}\text{tr}(\mathbf{V}_3 \mathbf{L} \mathbf{V}_3^T),
$
$$
\mathbf{V} = \begin{bmatrix}
\mathbf{V}_1 &  &   \\
 &\mathbf{V}_2&   \\
  &  & \mathbf{V}_3 \\
\end{bmatrix},
\quad
\mathbf{G} = 
\begin{bmatrix}
\mathbf{M}\\ 
\mathbf{I}\\ 
\mathbf{I}\\ 
\end{bmatrix},
\quad
\mathbf{B} = 
\begin{bmatrix}
-\mathbf{I} &  &  \\
  &-\mathbf{I}&  \\
&  & -\mathbf{I} \\
\end{bmatrix}
$$
The problem in \eqref{unmixing:graph-reg-ae-admm-1} can then be rewritten in an equivalent form as
\begin{equation}
    \label{unmixing:graph-reg-ae-admm-2}
    \begin{aligned}
        \underset{\mathbf{U},\mathbf{V}}{\text{minimize }} & \quad g(\mathbf{V})
        \\         
        \text{subject to } &  \quad \mathbf{GU} + \mathbf{BV} = \mathbf{0} 
   \end{aligned}
\end{equation}
The scaled augmented lagrangian $\mathcal{L}_\mu$ with parameter $\mu > 0$ and scaled dual variable $\mathbf{D}$ is then given as:
\begin{equation}
  \label{admm:lagrangian-ae}
  \mathcal{L}_{\mu}(\mathbf{U}, \mathbf{V}, \mathbf{D}) = g(\mathbf{V}) + \frac{\mu}{2} \|\mathbf{GU} + \mathbf{BV} - \mathbf{D}\|_F^2
\end{equation}
where
$$
\mathbf{D} = 
\begin{bmatrix}
\mathbf{D}_1 &  &  \\
  &\mathbf{D}_2&  \\
&  & \mathbf{D}_3 \\
\end{bmatrix} 
$$
ADMM aims to minimize scaled form of $\mathcal{L}_{\mu}$ by alternating minimizations with respect to $\mathbf{U}$, $\mathbf{V}$, and $\mathbf{D}$ by performing the following updates:
\begin{equation}
  \label{admm:ae-updates}
  \begin{aligned}
    \mathbf{U}^{(k+1)} &= \argmin_{\mathbf{U}}  \frac{\mu}{2} \|\mathbf{GU} + \mathbf{BV}^{(k)} - \mathbf{D}^{(k)}\|_F^2 \\
    \mathbf{V}^{(k+1)} &= \argmin_{\mathbf{V}} g(\mathbf{V}) + \frac{\mu}{2} \|\mathbf{GU}^{(k+1)} + \mathbf{BV} - \mathbf{D}^{(k)}\|_F^2 \\
    \mathbf{D}^{(k+1)} &= \mathbf{D}^{(k)} - \mathbf{GU}^{(k+1)} - \mathbf{BV}^{(k+1)} 
    \end{aligned}
\end{equation}
While the updates are in a simpler format, further work needs to be done to derive updates for $\mathbf{V}$. Looking at the $\|\mathbf{GU} + \mathbf{BV}^{(k)} - \mathbf{D}^{(k)}\|_F^2$ term in \eqref{admm:lagrangian-ae}, the structure of it's components give leeway to splitting the term into individual components, notably
\begin{equation*}
  \begin{aligned}
    \|\mathbf{GU} + \mathbf{BV} - \mathbf{D}\|_F^2 &= 
    \left\lVert
    \begin{bmatrix}
    \mathbf{MU} - \mathbf{V}_1 - \mathbf{D}_1 &  &  \\
      &\mathbf{U} - \mathbf{V}_2 - \mathbf{D}_2&  \\
    &  & \mathbf{U} - \mathbf{V}_3 - \mathbf{D}_3 \\
    \end{bmatrix}
    \right\rVert^2_F \\
    &= \|\mathbf{MU} - \mathbf{V}_1 - \mathbf{D}_1\|_F^2 +
       \|\mathbf{U} - \mathbf{V}_2 - \mathbf{D}_2\|_F^2 +
       \|\mathbf{U} - \mathbf{V}_3 - \mathbf{D}_3\|_F^2 
  \end{aligned}
\end{equation*}
Applying this expansion, the updates in \eqref{admm:ae-updates} can be rewritten. The $\mathbf{U}$ update becomes
\begin{equation}
  \label{admm:ae-updates-u}
  \begin{aligned}
    \mathbf{U}^{(k+1)} = \argmin_{\mathbf{U}}  & 
    \frac{\mu}{2} \|\mathbf{MU} - \mathbf{V}_1^{(k)} - \mathbf{D}_1^{(k)}\|_F^2 \\
    & + \frac{\mu}{2} \|\mathbf{U} - \mathbf{V}_2^{(k)} - \mathbf{D}_2^{(k)}\|_F^2 \\
    & + \frac{\mu}{2} \|\mathbf{U} - \mathbf{V}_3^{(k)} - \mathbf{D}_3^{(k)}\|_F^2 
  \end{aligned}
\end{equation}
Under the same expansion, the $\mathbf{V}$ update becomes
\begin{equation*}
  \begin{aligned}
    \mathbf{V}^{(k+1)} = \argmin_{\mathbf{V}}  &  \frac{1}{2} \|\mathbf{V}_1 - \mathbf{C} \|_F^2 + \chi_{\Delta}(\mathbf{V}_2) + \frac{\beta}{2}\text{tr}(\mathbf{V}_3\mathbf{L}\mathbf{V}_3^T) \\ 
    & + \frac{\mu}{2} \|\mathbf{MU}^{(k+1)} - \mathbf{V}_1 - \mathbf{D}_1^{(k)}\|_F^2 \\
    & + \frac{\mu}{2} \|\mathbf{U}^{(k+1)} - \mathbf{V}_2 - \mathbf{D}_2^{(k)}\|_F^2 \\
    & + \frac{\mu}{2} \|\mathbf{U}^{(k+1)} - \mathbf{V}_3 - \mathbf{D}_3^{(k)}\|_F^2 
  \end{aligned}
\end{equation*}
Furthermore, each component of the update for $\mathbf{V}$ can be split into individual updates for $\mathbf{V}_1$, $\mathbf{V}_2$ and $\mathbf{V}_3$,
\begin{equation}
  \label{admm:ae-updates-v}
  \begin{aligned}
    \mathbf{V}_1^{(k+1)} &= \argmin_{\mathbf{V}_1} \frac{1}{2} \|\mathbf{V}_1 - \mathbf{C} \|_F^2 + \frac{\mu}{2} \|\mathbf{MU}^{(k+1)} - \mathbf{V}_1 - \mathbf{D}_1^{(k)}\|_F^2 \\
    \mathbf{V}_2^{(k+1)} &= \argmin_{\mathbf{V}_2} \chi_{\Delta}(\mathbf{V}_2) + \frac{\mu}{2} \|\mathbf{U}^{(k+1)} - \mathbf{V}_2 - \mathbf{D}_2^{(k)}\|_F^2 \\
    \mathbf{V}_3^{(k+1)} &= \argmin_{\mathbf{V}_3} \frac{\beta}{2}\text{tr}(\mathbf{V}_3\mathbf{L}\mathbf{V}_3^T) + \frac{\mu}{2} \|\mathbf{U}^{(k+1)} - \mathbf{V}_3 - \mathbf{D}_3^{(k)}\|_F^2
  \end{aligned}
\end{equation}
Lastly, in similar fashion to $\mathbf{V}$, the $\mathbf{D}$ update in \eqref{admm:ae-updates} can also be split component wise:
\begin{equation}
  \label{admm:ae-updates-d}
  \begin{aligned}
    \mathbf{D}_1^{(k+1)} &= \mathbf{D}_1^{(k)} - \mathbf{MU}^{(k+1)} + \mathbf{V}_1^{(k+1)} \\
    \mathbf{D}_2^{(k+1)} &= \mathbf{D}_2^{(k)} - \mathbf{U}^{(k+1)} + \mathbf{V}_2^{(k+1)} \\
    \mathbf{D}_3^{(k+1)} &= \mathbf{D}_3^{(k)} - \mathbf{U}^{(k+1)} + \mathbf{V}_3^{(k+1)} 
  \end{aligned}
\end{equation}
Taking into account \eqref{admm:ae-updates-u}, \eqref{admm:ae-updates-v}, \eqref{admm:ae-updates-d}, the updates in \eqref{admm:ae-updates} can finally be rewritten in the expanded form as:
\begin{equation}
  \label{admm:ae-updates-final}
  \begin{aligned}
    \mathbf{U}^{(k+1)} & = \argmin_{\mathbf{U}}  
    \frac{\mu}{2} \|\mathbf{MU} - \mathbf{V}_1^{(k)} - \mathbf{D}_1^{(k)}\|_F^2  + \frac{\mu}{2} \|\mathbf{U} - \mathbf{V}_2^{(k)} - \mathbf{D}_2^{(k)}\|_F^2  + \frac{\mu}{2} \|\mathbf{U} - \mathbf{V}_3^{(k)} - \mathbf{D}_3^{(k)}\|_F^2 
    \\
    \mathbf{V}_1^{(k+1)} &= \argmin_{\mathbf{V}_1} \frac{1}{2} \|\mathbf{V}_1 - \mathbf{C} \|_F^2 + \frac{\mu}{2} \|\mathbf{MU}^{(k+1)} - \mathbf{V}_1 - \mathbf{D}_1^{(k)}\|_F^2 \\
    \mathbf{V}_2^{(k+1)} &= \argmin_{\mathbf{V}_2} \chi_{\Delta}(\mathbf{V}_2) + \frac{\mu}{2} \|\mathbf{U}^{(k+1)} - \mathbf{V}_2 - \mathbf{D}_2^{(k)}\|_F^2 \\
    \mathbf{V}_3^{(k+1)} &= \argmin_{\mathbf{V}_3} \frac{\beta}{2}\text{tr}(\mathbf{V}_3\mathbf{L}\mathbf{V}_3^T) + \frac{\mu}{2} \|\mathbf{U}^{(k+1)} - \mathbf{V}_3 - \mathbf{D}_3^{(k)}\|_F^2 \\
    \mathbf{D}_1^{(k+1)} &= \mathbf{D}_1^{(k)} - \mathbf{MU}^{(k+1)} + \mathbf{V}_1^{(k+1)} \\
    \mathbf{D}_2^{(k+1)} &= \mathbf{D}_2^{(k)} - \mathbf{U}^{(k+1)} + \mathbf{V}_2^{(k+1)} \\
    \mathbf{D}_3^{(k+1)} &= \mathbf{D}_3^{(k)} - \mathbf{U}^{(k+1)} + \mathbf{V}_3^{(k+1)} 
  \end{aligned}
\end{equation}

The updates for $\mathbf{U}$ and $\mathbf{V}_1$ have closed form solutions due to convexity and differentiability of the Frobenius norm. Both updates can derived by taking the partial derivatives with respect to the individual terms, setting them equal to $\mathbf{0}$, and solving accordingly. For the $\mathbf{U}$ update,
\begin{equation*}
  \begin{aligned}
    \mathbf{0} &= \frac{\partial}{\partial \mathbf{U}}\left[\frac{\mu}{2} \|\mathbf{MU} - \mathbf{V}_1 - \mathbf{D}_1\|_F^2  + \frac{\mu}{2} \|\mathbf{U} - \mathbf{V}_2 - \mathbf{D}_2\|_F^2  + \frac{\mu}{2} \|\mathbf{U} - \mathbf{V}_3 - \mathbf{D}_3\|_F^2\right]
    \\
    \mathbf{0} &= \mu \left(\mathbf{M}^T(\mathbf{MU}-\mathbf{V}_1-\mathbf{D}_1) + (\mathbf{U}-\mathbf{V}_2-\mathbf{D}_2) + (\mathbf{U}-\mathbf{V}_3-\mathbf{D}_3)\right)
    \\
    \mathbf{M}^T\mathbf{MU} + 2\mathbf{U} &= \mathbf{M}^T(\mathbf{V}_1+\mathbf{D}_1) + (\mathbf{V}_2+\mathbf{D}_2) + (\mathbf{V}_3+\mathbf{D}_3)
    \\
    \mathbf{U} &= (\mathbf{M}^T \mathbf{M} + 2\mathbf{I})^{-1}(\mathbf{M}^T(\mathbf{V}_1+\mathbf{D}_1) + (\mathbf{V}_2+\mathbf{D}_2) + (\mathbf{V}_3+\mathbf{D}_3)) 
  \end{aligned}
\end{equation*}
As $\mathbf{M}$ is known and unchanged, $(\mathbf{M}^T \mathbf{M} + 2\mathbf{I})^{-1}$ can be calculated and cached once for the entire runtime. For the $\mathbf{V}_1$ update,
$$
\begin{aligned}
   \mathbf{0} &= \frac{\partial}{\partial \mathbf{V}_1} \left[ \frac{1}{2}\|\mathbf{V}_1-\mathbf{C}\|_F^2 + \frac{\mu}{2} \|\mathbf{MU} - \mathbf{V}_1 - \mathbf{D}_1 \|_F^2 \right]
  \\
  \mathbf{0} &= (\mathbf{V}_1 - \mathbf{C}) + \mu(\mathbf{V}_1 - (\mathbf{MU} - \mathbf{D}_1))
  \\
  \mathbf{V}_1 &= \frac{1}{1+\mu} \left(\mathbf{C} + (\mathbf{MU} - \mathbf{D}_1)\right) 
\end{aligned}
$$

While the update for $\mathbf{V}_2$ does not have a closed form solution, it is important to note that the update $\mathbf{V}_2$ can be equivalently rewritten as:
$$
\mathbf{V}_2^{(k+1)} = \argmin_{\mathbf{V}_2 \in \Delta} \frac{\mu}{2} \|\mathbf{U}^{(k+1)} - \mathbf{V}_2 - \mathbf{D}_2^{(k)}\|_F^2 
$$
The update, in non-formulaic terms, requires finding $\mathbf{V}_2$ that minimizes $\frac{\mu}{2} \|\mathbf{U}^{(k+1)} - \mathbf{V}_2 - \mathbf{D}_2^{(k)}\|_F^2$, then projecting the solution onto $\Delta$. 
The non-projected minimum can be found in the same way as the updates for $\mathbf{V}$ and $\mathbf{U}$
\begin{equation*}
  \begin{aligned}
    \mathbf{0} &= \frac{\partial}{\partial \mathbf{V}_2} \left[\frac{\mu}{2} \|\mathbf{U} - \mathbf{V}_2 - \mathbf{D}_2\|_F^2\right]
    \\
    \mathbf{0} &= \mu(\mathbf{V}_2 - (\mathbf{MU} - \mathbf{D}_2))
    \\
    \mathbf{V}_2 &= \mathbf{MU} - \mathbf{D}_2 
  \end{aligned}
\end{equation*}
The orthogonal projection of a matrix $\mathbf{C}$ onto the convex and closed set $\Delta$ is defined as the finding the matrix $\tilde{\mathbf{C}} \in \Delta$ that minimizes the least-squares error between the two matrices. Multiple numerical methods exist for computing the projection (\cite{wang2013projection}). Formally, the projection can simply be written as follows.
\begin{equation*}
  \text{proj}_\Delta(\mathbf{C}) = \argmin_{\tilde{\mathbf{C}} \in \Delta} \|\tilde{\mathbf{C}} - \mathbf{C}\|_F^2 
\end{equation*}
Thus, applying the projection, the update for $\mathbf{V}_2$ is given as
\begin{equation*}
  \mathbf{V}_2 = \text{proj}_\Delta(\mathbf{MU} - \mathbf{D}_2) 
\end{equation*}

The approach for deriving update for $\mathbf{V}_3$ follows the same as $\mathbf{U}$ and $\mathbf{V}_1$ due to the convexity and differentiability of the regularization term $\text{tr}(\mathbf{V}_3\mathbf{L}\mathbf{V}_3^T)$. 
\begin{equation*}
  \begin{aligned}
    \mathbf{0} &= \frac{\partial}{\partial \mathbf{V}_3}\left[ \frac{\beta}{2} \text{tr}(\mathbf{V}_3\mathbf{L}\mathbf{V}_3^T) + \frac{\mu}{2} \|\mathbf{U} - \mathbf{V}_3 - \mathbf{D}_3\|_F^2  \right]
    \\
    \mathbf{0} &= \frac{\beta}{2} (\mathbf{V}_3\mathbf{L}^T + \mathbf{V}_3 \mathbf{L}) + \mu (\mathbf{V}_3 - (\mathbf{U} - \mathbf{D}_3))
    \\
    \mathbf{0} &= \beta \mathbf{V}_3 \mathbf{L} + \mu \mathbf{V}_3 - \mu (\mathbf{U} - \mathbf{D}_3)
    \\
    \mathbf{V}_3 \left(\mathbf{L} + \frac{\mu}{\beta} \mathbf{I}\right) &= \frac{\mu}{\beta}(\mathbf{U} - \mathbf{D}_3)
    \\
    \mathbf{V}_3 &= \frac{\mu}{\beta}(\mathbf{U} - \mathbf{D}_3) \left(\mathbf{L} + \frac{\mu}{\beta} \mathbf{I}\right)^{-1} 
  \end{aligned}
\end{equation*}
%Horn, Roger A.; Johnson, Charles R. (1985). Matrix Analysis. Cambridge University Press. ISBN 978-0-521-38632-6.
As $\mathbf{L}$ is a real valued, symmetric matrix, it can be eigendecomposed into the product $\mathbf{L} = \mathbf{S \Sigma S}^T$, where $\mathbf{S}$ is a matrix whose columns are the eigenvectors of $\mathbf{L}$, and $\mathbf{\Sigma}$ is a matrix whose diagonal elements are the eigenvalues of $\mathbf{L}$. Additionally, $\mathbf{S}$ is an orthogonal matrix, as such, $\mathbf{S}^T = \mathbf{S}^{-1}$ and $\mathbf{SS}^T = \mathbf{I}$. One important note to be made about the update step is that computing the inverse of the term $(\mathbf{L} + \mu / \beta \mathbf{I})$ is slower than computing the eigendecomposition $\mathbf{L}$ due to the naturally sparse definition of $\mathbf{L}$ and it's underlying distance matrix $\mathbf{W}_{\kappa}$. Using that information, a more efficient update can be performed by calculating the eigendecomposition $\mathbf{L} = \mathbf{S \Sigma S}^T$ and simplifying the update for $\mathbf{V}_3$ as follows:
\begin{equation*}
  \begin{aligned}
    \mathbf{V}_3 &= \frac{\mu}{\beta}(\mathbf{U} - \mathbf{D}_3) \left(\mathbf{L} + \frac{\mu}{\beta} \mathbf{I}\right)^{-1}
    \\
    \mathbf{V}_3 &= \frac{\mu}{\beta}(\mathbf{U} - \mathbf{D}_3) \left(\mathbf{S \Sigma S}^T + \frac{\mu}{\beta} \mathbf{SS}^T\right)^{-1}
    \\ 
    \mathbf{V}_3 &= \frac{\mu}{\beta}(\mathbf{U} - \mathbf{D}_3) 
    \left(\mathbf{S}\left(\mathbf{\Sigma} + \frac{\mu}{\beta} \mathbf{I}\right)  \mathbf{S}^T\right)^{-1}
    \\
    \mathbf{V}_3 &= \frac{\mu}{\beta}(\mathbf{U} - \mathbf{D}_3) \mathbf{S} \left(\mathbf{\Sigma} + \frac{\mu}{\beta} \mathbf{I}\right)^{-1} \mathbf{S}^T 
  \end{aligned}
\end{equation*}
The values along the diagonal in $\mathbf{\Sigma}$ are all non negative due to $\mathbf{L}$ having non negative eigenvalues. As such, the matrix $(\mathbf{\Sigma} + \mu / \beta \mathbf{I})$ is a diagonal matrix with the it's elements being all strictly positive. The inverse of $\left(\mathbf{\Sigma} + \mu / \beta \mathbf{I}\right)$ can be directly calculated as by taking the reciprocal of it's diagonal elements. As $\mathbf{L}$ is known at the onset of the algorithm and the parameters $\mu$ and $\beta$ do not change between iterations, $\mathbf{S} (\mathbf{\Sigma} + \mu / \beta \mathbf{I})^{-1} \mathbf{S}^T$ can be cached and reused across iterations.

The derived updates in \eqref{admm:ae-updates-final}, for parameters $\mu > 0$ and $\beta > 0$ are subsequently given as:
\begin{subequations}
  \begin{align}
    \mathbf{U}^{(k+1)} & = (\mathbf{M}^T \mathbf{M} + 2\mathbf{I})^{-1}\left(\mathbf{M}^T\left(\mathbf{V}_1^{(k)}+\mathbf{D}_1^{(k)}\right) + \left(\mathbf{V}_2^{(k)}+\mathbf{D}_2^{(k)}\right) + \left(\mathbf{V}_3^{(k)}+\mathbf{D}_3^{(k)}\right)\right) \label{unmixing:u} 
    \\
    \mathbf{V}_1^{(k+1)} &= \frac{1}{1+\mu} \left(\mathbf{C} + \left(\mathbf{MU}^{(k+1)} - \mathbf{D}_1^{(k)}\right)\right) \label{unmixing:v1} 
    \\
    \mathbf{V}_2^{(k+1)} &= \text{proj}_\Delta\left(\mathbf{MU}^{(k+1)} - \mathbf{D}_2^{(k)}\right) \label{unmixing:v2} 
    \\
    \mathbf{V}_3^{(k+1)} &= \frac{\mu}{\beta}\left(\mathbf{U}^{(k+1)} - \mathbf{D}_3^{(k)}\right) \mathbf{S} \left(\mathbf{\Sigma} + \frac{\mu}{\beta} \mathbf{I}\right)^{-1} \mathbf{S}^T \label{unmixing:v3} 
    \\
    \mathbf{D}_1^{(k+1)} &= \mathbf{D}_1^{(k)} - \mathbf{MU}^{(k+1)} + \mathbf{V}_1^{(k+1)} \label{unmixing:d1}  \\
    \mathbf{D}_2^{(k+1)} &= \mathbf{D}_2^{(k)} - \mathbf{U}^{(k+1)} + \mathbf{V}_2^{(k+1)} \label{unmixing:d2}  \\
    \mathbf{D}_3^{(k+1)} &= \mathbf{D}_3^{(k)} - \mathbf{U}^{(k+1)} + \mathbf{V}_3^{(k+1)} \label{unmixing:d3} 
  \end{align}
\end{subequations}
The algorithm is set to terminate when $ \|\mathbf{U}^{(k+1)} - \mathbf{U}^{(k)}\|_F/\|\mathbf{U}^{(k)}\|_F $ falls below a set tolerance $\epsilon$ or the algorithm reaches a maximum iterative index of $k_{\text{max}}$. After such point, $\mathbf{U}^{(k+1)}$ is given as the final result, representing the approximate solution of $\mathbf{A}$ to the minimization problem described in \eqref{unmixing:graph-reg-ae}. In it's entirety, the abundance estimation algorithm can be described with the following algorithm outline.

\begin{algorithm}[H]
  \label{Graph Regularized AE}
  \caption{Graph Regularized Abundance Estimation}
  \textbf{Input}: \\
  \quad  $\mathbf{C}$, $\mathbf{M}$, $\mathbf{W}_{\kappa}$, $\beta > 0$, $\mu > 0$, $k_{\text{max}} > 0$, $\epsilon > 0$.
  \\
  \textbf{Initialize:} 
  \\
  \quad Precompute and cache $\mathbf{S} (\mathbf{\Sigma} + \mu / \beta \mathbf{I})^{-1} \mathbf{S}^T$ and $(\mathbf{M}^T \mathbf{M} + 2\mathbf{I})^{-1}$\\
  \quad $\mathbf{U}^{(0)}  \in \Delta$ \\
  \quad $\mathbf{V}_1^{(0)} = \mathbf{MU}^{(0)}$ \\
  \quad $\mathbf{V}_2^{(0)} = \mathbf{U}^{(0)}$ \\
  \quad $\mathbf{V}_3^{(0)} = \mathbf{U}^{(0)}$ \\
  \quad $\mathbf{D}_1^{(0)} = \mathbf{0}$ \\
  \quad $\mathbf{D}_2^{(0)} = \mathbf{0}$ \\
  \quad $\mathbf{D}_3^{(0)} = \mathbf{0}$
  \\
  \textbf{For} $k = 0$ \text{to} $k_{\text{max}}$:\\
  \quad Update $\mathbf{U}^{(k+1)}$ according to \eqref{unmixing:u} \\
  \quad Update $\mathbf{V}_1^{(k+1)}$ according to \eqref{unmixing:v1} \\
  \quad Update $\mathbf{V}_2^{(k+1)}$ according to \eqref{unmixing:v2} \\
  \quad Update $\mathbf{V}_3^{(k+1)}$ according to \eqref{unmixing:v3} \\
  \quad Update $\mathbf{D}_1^{(k+1)}$ according to \eqref{unmixing:d1} \\
  \quad Update $\mathbf{D}_2^{(k+1)}$ according to \eqref{unmixing:d2} \\
  \quad Update $\mathbf{D}_3^{(k+1)}$ according to \eqref{unmixing:d3} \\
  \quad \textbf{Break if } $ \|\mathbf{U}^{(k+1)} - \mathbf{U}^{(k)}\|_F/\|\mathbf{U}^{(k)}\|_F < \epsilon$ \\
  \textbf{Output}:\\
  \quad Abundance Matrix $\mathbf{A} = \mathbf{U}$
\end{algorithm}

The ADMM-variant of the abundance estimation algorithm described above gives a efficient solution to in a relatively low number of iterations. Additional inquiry shows that the updates for $\mathbf{V}$ and $\mathbf{D}$ can be done completely in parallel, allowing for further performance optimization. In practical applications, $\beta$ is the only parameter to be tuned, corresponding to the strength of the spatial regularization, $\kappa$ is predefined in Section \ref{Algorithm NCuts}, as such it is not the focus of tuning in this step. 

% In Section \ref{AE}, the abundance estimation problem for a collection of pixels $\mathbf{X}$ given the endmember spectra matrix $\mathbf{M}$ was known was stated as follows:

\begin{equation*}
    A = \argmin_{A \in \mathbb{R}^{n_e \times n_p}} \frac{1}{2}\|\mathbf{MA} - \mathbf{X}\|_F^2 + \chi_\Delta(\mathbf{A}) + J(\mathbf{A}).
\end{equation*}
The goal of this section is to demonstrate how this problem can be equivalently represented in a form where the alternating direction method of multipliers technique can be applied. The abundance estimation problem when one or more than one regularization terms are added belongs to a class of problems called global consensus optimization problems shown in Boyd et al. [REF]

Operating under the assumption that $J$ is a convex function, since for nonconvex choices of $J$, convergence is not guarenteed. The approach to transforming \eqref{ae:ae-min-1} is to first introduce matrices $\mathbf{U} \in \mathbb{R}^{n_e \times n_p}$, $\mathbf{V}_1 \in \mathbb{R}^{n_b \times n_p}$, $\mathbf{V}_2 \in \mathbb{R}^{n_e \times n_p}$, and $\mathbf{V}_{3} \in \mathbb{R}^{n_e \times n_p}$ and rewrite as follows:
\begin{equation}
    \label{ae:equivalent-admm-1}
    \begin{aligned}
        \underset{\mathbf{U}, \mathbf{V}_1, \mathbf{V}_2, \mathbf{V}_3}{\text{minimize }} & \quad \frac{1}{2} \|\mathbf{V}_1 - X \|_F^2 + \chi_{\Delta}(\mathbf{V}_2) + J(\mathbf{V}_3) 
        \\         
        \text{subject to } &  \quad \mathbf{V}_1 = \mathbf{MU} \\
        & \quad \mathbf{V}_2 = \mathbf{U} \\
        & \quad \mathbf{V}_{3} = \mathbf{U}
   \end{aligned}
\end{equation}
where
$$g(\mathbf{V}) = \frac{1}{2} \|\mathbf{V}_1 - X \|_F^2 + \chi_{\Delta}(\mathbf{V}_2) + J(\mathbf{V}_3)$$
$$
\mathbf{V} = \begin{bmatrix}
\mathbf{V}_1 &  &   \\
 &\mathbf{V}_2&   \\
  &  & \mathbf{V}_3 \\
\end{bmatrix},
\quad
\mathbf{G} = 
\begin{bmatrix}
\mathbf{M}\\ 
\mathbf{I}\\ 
\mathbf{I}\\ 
\end{bmatrix},
\quad
\mathbf{B} = 
\begin{bmatrix}
-\mathbf{I} &  &  \\
  &-\mathbf{I}&  \\
&  & -\mathbf{I} \\
\end{bmatrix}.
$$
The problem depicted in \eqref{ae:equivalent-admm-1} can then be rewritten in equivalent form:
\begin{equation}
    \label{ae:equivalent-admm-2}
    \begin{aligned}
        \underset{\mathbf{U},\mathbf{V}}{\text{minimize }} & \quad g(\mathbf{V})
        \\         
        \text{subject to } &  \quad \mathbf{GU} + \mathbf{BV} = \mathbf{0}
   \end{aligned}
\end{equation}
The scaled augmented lagrangian $\mathcal{L}_\mu$ with parameter $\mu > 0$ and scaled dual variable $\mathbf{D}$ is then given as:
\begin{equation}
  \label{admm:lagrangian-ae}
  \mathcal{L}_{\mu}(\mathbf{U}, \mathbf{V}, \mathbf{D}) = g(\mathbf{V}) + \frac{\mu}{2} \|\mathbf{GU} + \mathbf{BV} - \mathbf{D}\|_F^2
\end{equation}
where
$$
\mathbf{D} = 
\begin{bmatrix}
\mathbf{D}_1 &  &  \\
  &\mathbf{D}_2&  \\
&  & \mathbf{D}_3 \\
\end{bmatrix}.
$$
ADMM aims to minimize scaled form of $\mathcal{L}_{\mu}$ by alternating minimizations with respect to $\mathbf{U}$, $\mathbf{V}$, and $\mathbf{D}$ by performing the following updates:
\begin{equation}
  \label{admm:ae-updates}
  \begin{aligned}
    \mathbf{U}^{(k+1)} &= \argmin_{\mathbf{U}}  \frac{\mu}{2} \|\mathbf{GU} + \mathbf{BV}^{(k)} - \mathbf{D}^{(k)}\|_F^2 \\
    \mathbf{V}^{(k+1)} &= \argmin_{\mathbf{V}} g(\mathbf{V}) + \frac{\mu}{2} \|\mathbf{GU}^{(k+1)} + \mathbf{BV} - \mathbf{D}^{(k)}\|_F^2 \\
    \mathbf{D}^{(k+1)} &= \mathbf{D}^{(k)} - \mathbf{GU}^{(k+1)} - \mathbf{BV}^{(k+1)}.
    \end{aligned}
\end{equation}

While the updates are in a simpler format, further work needs to be done to derive updates for $\mathbf{V}$. Looking at the $\|\mathbf{GU} + \mathbf{BV}^{(k)} - \mathbf{D}^{(k)}\|_F^2$ term in \eqref{admm:lagrangian-ae}, the structure of it's components give leeway to splitting the term into individual components, notably
\begin{equation*}
  \begin{aligned}
    \|\mathbf{GU} + \mathbf{BV} - \mathbf{D}\|_F^2 &= 
    \left\lVert
    \begin{bmatrix}
    \mathbf{MU} - \mathbf{V}_1 - \mathbf{D}_1 &  &  \\
      &\mathbf{U} - \mathbf{V}_2 - \mathbf{D}_2&  \\
    &  & \mathbf{U} - \mathbf{V}_3 - \mathbf{D}_3 \\
    \end{bmatrix}
    \right\rVert^2_F \\
    &= \|\mathbf{MU} - \mathbf{V}_1 - \mathbf{D}_1\|_F^2 +
       \|\mathbf{U} - \mathbf{V}_2 - \mathbf{D}_2\|_F^2 +
       \|\mathbf{U} - \mathbf{V}_3 - \mathbf{D}_3\|_F^2.
  \end{aligned}
\end{equation*}
Applying this expansion, the updates in \eqref{admm:ae-updates} can be rewritten. The $\mathbf{U}$ update becomes
\begin{equation}
  \label{admm:ae-updates-u}
  \begin{aligned}
    \mathbf{U}^{(k+1)} = \argmin_{\mathbf{U}}  & 
    \frac{\mu}{2} \|\mathbf{MU} - \mathbf{V}_1^{(k)} - \mathbf{D}_1^{(k)}\|_F^2 \\
    & + \frac{\mu}{2} \|\mathbf{U} - \mathbf{V}_2^{(k)} - \mathbf{D}_2^{(k)}\|_F^2 \\
    & + \frac{\mu}{2} \|\mathbf{U} - \mathbf{V}_3^{(k)} - \mathbf{D}_3^{(k)}\|_F^2.
  \end{aligned}
\end{equation}
Under the same expansion, the $\mathbf{V}$ update becomes
\begin{equation*}
  \begin{aligned}
    \mathbf{V}^{(k+1)} = \argmin_{\mathbf{V}}  &  \frac{1}{2} \|\mathbf{V}_1 - X \|_F^2 + \chi_{\Delta}(\mathbf{V}_2) + J(\mathbf{V}_3) \\ 
    & + \frac{\mu}{2} \|\mathbf{MU}^{(k+1)} - \mathbf{V}_1 - \mathbf{D}_1^{(k)}\|_F^2 \\
    & + \frac{\mu}{2} \|\mathbf{U}^{(k+1)} - \mathbf{V}_2 - \mathbf{D}_2^{(k)}\|_F^2 \\
    & + \frac{\mu}{2} \|\mathbf{U}^{(k+1)} - \mathbf{V}_3 - \mathbf{D}_3^{(k)}\|_F^2.
  \end{aligned}
\end{equation*}
Furthermore, each component of the update for $\mathbf{V}$ can be split into individual updates for $\mathbf{V}_1$, $\mathbf{V}_2$ and $\mathbf{V}_3$:
\begin{equation}
  \label{admm:ae-updates-v}
  \begin{aligned}
    \mathbf{V}_1^{(k+1)} &= \argmin_{\mathbf{V}_1} \frac{1}{2} \|\mathbf{V}_1 - X \|_F^2 + \frac{\mu}{2} \|\mathbf{MU}^{(k+1)} - \mathbf{V}_1 - \mathbf{D}_1^{(k)}\|_F^2 \\
    \mathbf{V}_2^{(k+1)} &= \argmin_{\mathbf{V}_2} \chi_{\Delta}(\mathbf{V}_2) + \frac{\mu}{2} \|\mathbf{U}^{(k+1)} - \mathbf{V}_2 - \mathbf{D}_2^{(k)}\|_F^2 \\
    \mathbf{V}_3^{(k+1)} &= \argmin_{\mathbf{V}_3} J(\mathbf{V}_3) + \frac{\mu}{2} \|\mathbf{U}^{(k+1)} - \mathbf{V}_3 - \mathbf{D}_3^{(k)}\|_F^2
  \end{aligned}
\end{equation}
Lastly, in similar fashion to $\mathbf{V}$, the $\mathbf{D}$ update in \eqref{admm:ae-updates} can also be split component wise:
\begin{equation}
  \label{admm:ae-updates-d}
  \begin{aligned}
    \mathbf{D}_1^{(k+1)} &= \mathbf{D}_1^{(k)} - \mathbf{MU}^{(k+1)} + \mathbf{V}_1^{(k+1)} \\
    \mathbf{D}_2^{(k+1)} &= \mathbf{D}_2^{(k)} - \mathbf{U}^{(k+1)} + \mathbf{V}_2^{(k+1)} \\
    \mathbf{D}_3^{(k+1)} &= \mathbf{D}_3^{(k)} - \mathbf{U}^{(k+1)} + \mathbf{V}_3^{(k+1)}.
  \end{aligned}
\end{equation}
Taking into account \eqref{admm:ae-updates-u}, \eqref{admm:ae-updates-v}, \eqref{admm:ae-updates-d}, the updates in \eqref{admm:ae-updates} can finally be rewritten in the expanded form as:
\begin{equation}
  \label{admm:ae-updates-final}
  \begin{aligned}
    \mathbf{U}^{(k+1)} & = \argmin_{\mathbf{U}}  
    \frac{\mu}{2} \|\mathbf{MU} - \mathbf{V}_1^{(k)} - \mathbf{D}_1^{(k)}\|_F^2  + \frac{\mu}{2} \|\mathbf{U} - \mathbf{V}_2^{(k)} - \mathbf{D}_2^{(k)}\|_F^2  + \frac{\mu}{2} \|\mathbf{U} - \mathbf{V}_3^{(k)} - \mathbf{D}_3^{(k)}\|_F^2 
    \\
    \mathbf{V}_1^{(k+1)} &= \argmin_{\mathbf{V}_1} \frac{1}{2} \|\mathbf{V}_1 - X \|_F^2 + \frac{\mu}{2} \|\mathbf{MU}^{(k+1)} - \mathbf{V}_1 - \mathbf{D}_1^{(k)}\|_F^2 \\
    \mathbf{V}_2^{(k+1)} &= \argmin_{\mathbf{V}_2} \chi_{\Delta}(\mathbf{V}_2) + \frac{\mu}{2} \|\mathbf{U}^{(k+1)} - \mathbf{V}_2 - \mathbf{D}_2^{(k)}\|_F^2 \\
    \mathbf{V}_3^{(k+1)} &= \argmin_{\mathbf{V}_3} J(\mathbf{V}_3) + \frac{\mu}{2} \|\mathbf{U}^{(k+1)} - \mathbf{V}_3 - \mathbf{D}_3^{(k)}\|_F^2 \\
    \mathbf{D}_1^{(k+1)} &= \mathbf{D}_1^{(k)} - \mathbf{MU}^{(k+1)} + \mathbf{V}_1^{(k+1)} \\
    \mathbf{D}_2^{(k+1)} &= \mathbf{D}_2^{(k)} - \mathbf{U}^{(k+1)} + \mathbf{V}_2^{(k+1)} \\
    \mathbf{D}_3^{(k+1)} &= \mathbf{D}_3^{(k)} - \mathbf{U}^{(k+1)} + \mathbf{V}_3^{(k+1)}.
  \end{aligned}
\end{equation}

The updates for $\mathbf{U}$ and $\mathbf{V}_1$ have closed form solutions due to convexity and differentiability of the Frobenius norm [REF]. Both updates can derived by taking the first partial derivatives with respect to the individual terms, setting it equal to $\mathbf{0}$, and solving accordingly. For the $\mathbf{U}$ update,
$$
  \begin{aligned}
    0 &= \frac{\partial}{\partial \mathbf{U}}\left[\frac{\mu}{2} \|\mathbf{MU} - \mathbf{V}_1 - \mathbf{D}_1\|_F^2  + \frac{\mu}{2} \|\mathbf{U} - \mathbf{V}_2 - \mathbf{D}_2\|_F^2  + \frac{\mu}{2} \|\mathbf{U} - \mathbf{V}_3 - \mathbf{D}_3\|_F^2\right]
    \\
    0 &= \mu \left(\mathbf{M}^T(\mathbf{MU}-\mathbf{V}_1-\mathbf{D}_1) + (\mathbf{U}-\mathbf{V}_2-\mathbf{D}_2) + (\mathbf{U}-\mathbf{V}_3-\mathbf{D}_3)\right)
    \\
    \mathbf{M}^T\mathbf{MU} + 2\mathbf{U} &= \mathbf{M}^T(\mathbf{V}_1+\mathbf{D}_1) + (\mathbf{V}_2+\mathbf{D}_2) + (\mathbf{V}_3+\mathbf{D}_3)
    \\
    \mathbf{U} &= (\mathbf{M}^T \mathbf{M} + 2\mathbf{I})^{-1}(\mathbf{M}^T(\mathbf{V}_1+\mathbf{D}_1) + (\mathbf{V}_2+\mathbf{D}_2) + (\mathbf{V}_3+\mathbf{D}_3)).
  \end{aligned}
$$
As $\mathbf{M}$ is known, $(\mathbf{M}^T \mathbf{M} + 2\mathbf{I})^{-1}$ can be calculated once for the entire program. For the $\mathbf{V}$ update,
$$
\begin{aligned}
  0 &= \frac{\partial}{\partial \mathbf{V}_1} \left[ \frac{1}{2}\|\mathbf{V}_1-\mathbf{X}\|_F^2 + \frac{\mu}{2} \|\mathbf{MU} - \mathbf{V}_1 - \mathbf{D}_1 \|_F^2 \right]
  \\
  0 &= (\mathbf{V}_1 - \mathbf{X}) + \mu(\mathbf{V}_1 - (\mathbf{MU} - \mathbf{D}_1))
  \\
  \mathbf{V}_1 &= \frac{1}{1+\mu} \left(\mathbf{X} + (\mathbf{MU} - \mathbf{D}_1)\right).
\end{aligned}
% 0 &= \frac{\partial}{\partial V_1} \left[ \frac{1}{2}\|V_1-X\|_F^2 + \frac{\tau}{2} \|MU - V_1 - D_1 \|_F^2 \right] \\
% 0 &= (V_1 - X) + \tau(V_1 - (MU - D_1)) \\
% V_1 &= \frac{1}{1+\tau} \left(X + (MU - D_1)\right)
% \end{aligned}
$$
% While the update for $\mathbf{V}_2$ does not have a closed form solution, the convexity of $\Delta$ allows for the use of an alternating projection based method for finding an approximate solution to the update. $\Delta$ itself is the intersection of two convex sets $\mathbb{R}^{n_e \times n_p}_+$ and $\{ \mathbf{A} \in \mathbb{R}^{n_e \times n_p} \mid \mathbf{1}_{n_e}^T \mathbf{A} = \mathbf{1}_{n_p}\}$. It is important to note that the update $\mathbf{V}_2$ can be rewritten as 

While the update for $\mathbf{V}_2$ does not have a closed form solution, it is important to note that the update $\mathbf{V}_2$ can be equivalently rewritten as:
$$
\mathbf{V}_2^{(k+1)} = \argmin_{\mathbf{V}_2 \in \Delta} \frac{\mu}{2} \|\mathbf{U}^{(k+1)} - \mathbf{V}_2 - \mathbf{D}_2^{(k)}\|_F^2.
$$
The update, in non-formulaic terms, requires finding $\mathbf{V}_2$ that minimizes $\frac{\mu}{2} \|\mathbf{U}^{(k+1)} - \mathbf{V}_2 - \mathbf{D}_2^{(k)}\|_F^2$, then projecting the solution onto $\Delta$. 
The non-projected minimum can be found in the same way as the updates for $\mathbf{V}$ and $\mathbf{U}$
\begin{equation*}
  \begin{aligned}
    0 &= \frac{\partial}{\partial \mathbf{V}_2} \left[\frac{\mu}{2} \|\mathbf{U} - \mathbf{V}_2 - \mathbf{D}_2\|_F^2\right]
    \\
    0 &= \mu(\mathbf{V}_2 - (\mathbf{MU} - \mathbf{D}_2))
    \\
    \mathbf{V}_2 &= \mathbf{MU} - \mathbf{D}_2
  \end{aligned}
\end{equation*}
The orthogonal projection of a matrix $\mathbf{X}$ onto $\Delta$ is defined as the finding the matrix $\tilde{\mathbf{X}} \in \Delta$ that minimizes the least-squares error between the two matrices. The convexity of $\Delta$ and the additional property that $\Delta$ is closed ensures that the projection is unique. Multiple numerical methods exist for computing the projection [REF]. Formally,
\begin{equation*}
  \text{proj}_\Delta(\mathbf{X}) = \argmin_{\tilde{\mathbf{X}} \in \Delta} \|\tilde{\mathbf{X}} - \mathbf{X}\|_F^2.
\end{equation*}
Thus, applying the projection, the update for $\mathbf{V}_2$ is given as
\begin{equation*}
  \mathbf{V}_2 = \text{proj}_\Delta(\mathbf{MU} - \mathbf{D}_2).
\end{equation*}

After deriving solutions for $\mathbf{V}_1$, $\mathbf{V}_2$, $\mathbf{U}$, the updates in \eqref{admm:ae-updates} can be written as:
\begin{equation}
  \label{admm:ae-updates-final-2}
  \begin{aligned}
    \mathbf{U}^{(k+1)} & = (\mathbf{M}^T \mathbf{M} + 2\mathbf{I})^{-1}(\mathbf{M}^T(\mathbf{V}_1^{(k)}+\mathbf{D}_1^{(k)}) + (\mathbf{V}_2^{(k)}+\mathbf{D}_2^{(k)}) + (\mathbf{V}_3^{(k)}+\mathbf{D}_3^{(k)})).
    \\
    \mathbf{V}_1^{(k+1)} &= \frac{1}{1+\mu} \left(\mathbf{X} + (\mathbf{MU}^{(k+1)} - \mathbf{D}_1^{(k)})\right) 
    \\
    \mathbf{V}_2^{(k+1)} &= \text{proj}_\Delta(\mathbf{MU}^{(k+1)} - \mathbf{D}_2^{(k)}) 
    \\
    \mathbf{V}_3^{(k+1)} &= \argmin_{\mathbf{V}_3} J(\mathbf{V}_3) + \frac{\mu}{2} \|\mathbf{U}^{(k+1)} - \mathbf{V}_3 - \mathbf{D}_3^{(k)}\|_F^2 
    \\
    \mathbf{D}_1^{(k+1)} &= \mathbf{D}_1^{(k)} - \mathbf{MU}^{(k+1)} + \mathbf{V}_1^{(k+1)} 
    \\
    \mathbf{D}_2^{(k+1)} &= \mathbf{D}_2^{(k)} - \mathbf{U}^{(k+1)} + \mathbf{V}_2^{(k+1)} 
    \\
    \mathbf{D}_3^{(k+1)} &= \mathbf{D}_3^{(k)} - \mathbf{U}^{(k+1)} + \mathbf{V}_3^{(k+1)}.
  \end{aligned}
\end{equation}

Leaving the update for $\mathbf{V_3}$ as the only minimization problem for practicioners to derive the update for. Many variations of the abundance estimation problem exist where one or more regularization terms on $\mathbf{A}$ are added for further control over the final solution. This technique can be further extended to consider $m$ regularization terms $J_1, \dots, J_m$, in which the result is $m+2$ additional updates for the subproblems arising from splitting $\mathbf{V}$ and $m+2$ additional updates for the subproblems arising from splitting $\mathbf{D}$. The alternating direction method of multipliers technique is most useful in situations where multiple regularization terms exist in the loss function and no closed form solution exist, demonstrating it's superiority in the field of hyperspectral image analysis.


\subsection{Feature Vector Creation}\label{Algorithm FV}
The abundance estimation problem described in Section \ref{Algorithm Unmixing} outputs a matrix $ \mathbf{A} \in \mathbb{R}_+^{n_e \times n_p}$ whose columns $\mathbf{a}_i$ hold the abundance values for the superpixel $\mathbf{c}_i$ with respect to the mean spectral signatures. The initial partitioning of the set of superpixels performed in Section \ref{Algorithm NCuts} often provides an adequate solution, however, there is always hope to further refine the segmentation (\cite{SSBD}). In the effort to do so, a new superpixel feature vector is created by concatenating the original spectral features and the abundance estimation results
\begin{equation}
    \label{Feature Vector Creation}
    \tilde{\mathbf{c}}_i = \mathbf{c}_i \oplus \mathbf{a}_i
\end{equation}

With the new construction of $\tilde{\mathbf{c}}$, the segmentation described in Section \ref{Algorithm NCuts} is repeated with the same set of parameters $\sigma$ and $\kappa$. The overall intuition behind this final step is to combine these additional, rich abundance features and the original spectral information to produce a more accurate segmentation. This proves useful along the boundaries between regions, where the creation of superpixels leads to mixed pixels. The added abundance information aids in adequately segmenting these mixed regions. The further refinement through the inclusion of the abundance information allows stronger spatial coherence in the clusters, due to the use of the graph regularization in \eqref{unmixing:graph-reg-ae}.


% \clearpage
\subsection{Algorithm Overview}\label{Algorithm Overview}
In summary, the algorithm can be described as follows

\begin{algorithm}[H]
    \caption{Adaptive Superpixel Cuts for Hyperspectral Images}
    \textbf{Input}: \\
    \quad Hyperspectral Image $\mathbf{X} \in \mathbb{R}_+^{n_x \times n_y \times n_\lambda}$. \\
    \quad Superpixel Parameters: $n_s$, $m$\\
    \quad Segmentation Parameters: $n_e$, $\sigma$ , $\kappa$ \\
    \quad Abundance Estimation Parameters: $\mu$ , $\beta$
    \\
    \textbf{Preprocessing:}\\ \quad Create Normalized Hyperspectral Image $\hat{\mathbf{X}}$ according to \eqref{alg:normalization}.
    \\
    \textbf{Superpixel Creation:}\\ \quad Generate superpixel matrix $\mathbf{C} \in \mathbb{R}_+^{n_\lambda \times n_s}$ according to Algorithm \ref{HSI SLIC} with parameters $n_s$, $m$.
    \\
    \textbf{Spatial Spectral Segmentation:}\\ \quad Perform an initial segmentation of the columns of the superpixel matrix $\mathbf{C}$ into $n_e$ partitions and form the spectra matrix $\mathbf{M} \in \mathbf{R}_+^{n_\lambda \times n_e}$ according to Algorithm \ref{Spatial Spectral Segmentation} with parameters $n_e$, $\sigma$, $\kappa$.
    \\
    \textbf{Abundance Estimation:}\\ \quad Perform abundance estimation to the columns of the superpixel matrix $\mathbf{C}$ relative to the spectra matrix $\mathbf{M}$, obtaining abudance matrix $A \in \mathbb{R}_+^{n_e \times n_s} $ according to Algorithm \ref{Graph Regularized AE} with parameters $\mu$ , $\beta$.
    \\
    \textbf{Feature Vector Creation}\\ \quad Form the feature matrix $\tilde{\mathbf{C}} \in \mathbb{R}_+^{(n_\lambda + n_e) \times n_s}$ according to \eqref{Feature Vector Creation}
    \\
    \textbf{Spatial Spectral Segmentation:}\\ \quad Perform the final segmentation of the columns of the superpixel matrix $\tilde{\mathbf{C}}$ into $n_e$ partitions according to Algorithm \ref{Spatial Spectral Segmentation} with parameters $n_e$, $\sigma$, $\kappa$.
    \\
    \textbf{Output}:\\
    \quad Label vector $\mathbf{v}$, where $v_i \in {1,2, \cdots, n_e}$, corresponding to the final segmentation of the superpixels.
  \end{algorithm}

The superpixel parameters $n_s$ and $m$ are one time selections based off the requirements of the practioners, as $n_s$ determines the runtime of the algorithm. If $n_s$ is set too high, the more the superpixels resemble the original image itself, giving zero benefit to using a superpixel approach. Careful and reasonable selection of $n_s$ and $m$ should be done based off the requirements of the specific analysis to be done. In practice, setting $n_s$ such that $\frac{n_p}{n_s} \approx 16$ allows adequate results without extensive tuning of $m$. In similar fashion, the abundance estimation parameters $\mu$ and $\beta$ are one time, global selections. As mentioned in Section \ref{ADMM Intro}, the ADMM method provides modest accuracy solutions in a relatively low number of operations. In practice, the results of the algorithm are largerly insensitive to selection of $\mu$. In a similar manner, $\beta$ serves the purpose of reducing relative variations in abundance estimates between nearby superpixels, which aids in instances where two regions share similar spectral characteristics but are spatially distinct. 

In application, the choice of parameters primarily focuses on on the segmentation parameters $n_e$, $\sigma$ , $\kappa$ as they dictate the initial and final segmentation of the image itself. Selection should be done to ensure an informative intial segmentation, then allow the algorithm to further refine it for the final segmentation according to the specific domain requirements.

\clearpage
% #############################################
% 
% 
% 
% 
% 
% #############################################
\section{Experimental Results}
Having presented the details of the proposed approach, this section aims to document the experimental evaluation of its performance with respect to well studied hyperpsectral datasets in the remote sensing field. Additionally, comparison in results is made to two popular segmentation algorithms. This section will demonstrate the ability of our method to acheive accurate segmentation, with a focus on its advantages in handling mixed pixels and spatially seperated materials.

% All experiements were performed on an 8-core AMD Ryzen 9 6900HS with 32GB of 4800MHz RAM. The implementation of the algorithm was done in Python 3.11 using the numpy and scipy libraries. 

\subsection{Evaluation Datasets}
There exist two popular datasets in the remote sensing community are the Samson and Salinas dataset. 

% https://aviris.jpl.nasa.gov/
The Salinas is a 512-by-217 hyperspectral image collected by the 224-band AVIRIS sensor over Salinas Valley, California, and is characterized by high spatial resolution (3.7-meter pixels) with spectral wavelengths from 400 to 2500 nanometers. 20 spectral bands were removed from the image corresponding to high-noise channels, resulting in a total of 204 captured spectral bands. The image is comprised of 16 classifications based on various vegetation and vineyard types. [REF] 
\begin{figure}[H]
    \centering
    \includegraphics[width=10cm]{salinas_overview.png}  % Adjust width and filename
    \caption{Colorized Salinas \& Ground Truth Labels with Subsets A (White Box) \& B (Red Box)}
    \label{salinas-borders}  % Optional label for referencing
  \end{figure}

The Salinas dataset is commonly used in hyperspectral classification evaluation due to the availibility of ground truth labels within the image. Due to similar spectral characteristics of multiple vegetation types in the image, this dataset is optimal to test the efficacy of the proposed algorithm in segmentation tasks. There are two subsets of the Salinas commonly used for evaluation. Salinas A is a 86-by-83 pixel subset of the Salinas image beginning from pixel index $(270, 0)$ with 6 classifications. Salinas B is a 55-by-150 pixel subset of the Salinas image beginning from pixel index $(35, 25)$ with 5 classifications. Both subsections provide an adequate mix of vegetation types.
\begin{figure}[H]
    \centering
    \includegraphics[width=8cm]{salinas-a.png}  % Adjust width and filename
    \caption{Greyscale Salinas-A \& Ground Truth Labels}
    % \label{salinas-a}  % Optional label for referencing
    \includegraphics[width=8cm]{salinas-b.png}  % Adjust width and filename
    \caption{Greyscale Salinas-B \& Ground Truth Labels}
    \label{salinas-ab}  % Optional label for referencing
  \end{figure}

In similar fashion, Samson is a 952-by-952 hyperspectral image captured by 156-band SAMSON sensor over Elkhorn, California. The image covers a spectral range of 400nm to 900nm with a bandwidth of 3.2nm. This image does not include any ground truth classification, however subsets of the image are created comprised of soil, trees, and water. [REF]
\begin{figure}[H]
  \centering
  \includegraphics[width=10cm]{samson_full.png}  % Adjust width and filename
  \caption{Colorized Samson \& with Subsets A (Red), B (Blue), C (Green)}
  \label{samson}  % Optional label for referencing
\end{figure}

The Samson-A, Samson-B, Samson-C subsets are all 95-by-95 hyperspectral images of regions originating from pixel indices (332, 252), (93, 232) and (345, 545) respectiely. The subsets are comprised of the shore lines of where there are many mixed pixel measurements occur due the mixing of materials and complex forms created by the distribution of materials.
\begin{figure}[H]
  \centering
  \includegraphics[width=15cm]{samsonabc.png}  % Adjust width and filename
  \caption{Greyscale Samson Subsets}
  \label{samsonabc}  % Optional label for referencing
\end{figure}




\clearpage
\subsection{Algorithm Evaluation}
All experiements were performed on an 8-core AMD Ryzen 9 6900HS with 32GB of 4800MHz RAM. The implementation of the algorithm was done in Python 3.11 using the numpy and scipy libraries. 
\subsubsection{Quantitative Evaluation on Salinas}
Our proposed segmentation algorithm was applied to the Salinas dataset and its performance was evaluated using established evaluation metrics like Overall Accuracy (OA), Average Segmentation Accuracy (ASA), and Intersection over Union (IoU) for individual classes. This evaluation will allow us to assess the effectiveness of our algorithm in segmenting the Salinas scene, particularly focusing on its ability to accurately identify and delineate the various material classes present within the dataset.

For Salinas-A, prior knowledge about the scene confirms the aim is to segment the image into $n_e = 6$ segments corresponding to the following vegetation classifications in the image. One important note is that Label 2 - Corn-senesced green weed is commonly understood to be incorrectly label, this is due to the presence of two differing spectral signatures in endmember. As such, the aim is to then segment the image in $n_e = 7$ segments corresponding to this new spectral distinction within Label 2. For the sake of evaluation against the known labels, segmentations created by the algorithm within this region will be joined into one label.
\begin{table}[H]
    % \caption{Groundtruth classes for the Salinas-A scene and their respective samples number}
    \centering
    \label{tab:salinas_classes}
    \begin{tabular}{|c|c|c|}
    \hline
    \textbf{Label} & \textbf{Class} & \textbf{Samples} \\
    \hline
    1 & Broccoli green weeds 1 & 391 \\
    2 & Corn-senesced green weeds & 1343 \\
    3 &Lettuce romaine 4wk & 616 \\
    4 & Lettuce romaine 5wk & 1525 \\
    5 & Lettuce romaine 6wk & 674 \\
    6 &Lettuce romaine 7wk & 799 \\
    \hline
    \end{tabular}
    \caption{Groundtruth classes for the Salinas-A scene and their respective samples number}
\end{table}
\begin{figure}[h]
    \centering
    \includegraphics[width=10cm]{salinas-a-error.png}  % Adjust width and filename
    \caption{Salinas-A Ground Truth Labels}
    \label{salina-a}  % Optional label for referencing
  \end{figure}
\clearpage

Using an initial selection of $n_s = 420$ superpixels with shape parameter $m=2$ and unmixing parameters $\mu = 1$ and $\beta = 0.0025$, a grid search was applied to $\sigma$ and $\kappa$ such that $\sigma \in [0.1, 0.001]$ and $\kappa \in [15, 40]$, to which $\sigma = 0.0025$ and $\kappa = 20$ produced the most optimal results with an overall accuracy of $0.993$.
\begin{figure}[H]
    \centering
    \includegraphics[width=10cm]{salinas-a-labelled.png}  % Adjust width and filename
    \caption{Salinas-A Algorithm Results}
    \label{salina-a-results}  % Optional label for referencing
\end{figure}
In the case of this specific segmentation, Label $2$ is comprised of two individual segments were combined to match with the comparison to the original Corn-senesced green weed class. The algorithm itself performs well across all classes, with high individual segment accuracy scores as well as IoU scores, demonstrating it's ability to perform segmentation in scenarios where similar spectral features are shared among materials and materials span across the image.
\begin{table}[H]
    \centering
    \label{tab:salinas_cfm}
    \begin{tabular}{|c|cccccc|c|c|}
        \hline
         & \textbf{1} & \textbf{2} & \textbf{3} & \textbf{4} & \textbf{5} & \textbf{6} & \textbf{Accuracy} & \textbf{IoU} \\ \hline
        1      &  390    &  0    &  0    &  10    &  0    &  0    &  0.997 & 0.997  \\ 
        2      &  0    & 1343  &  0    &  0    &  0    &  0     &  1.000   & 0.987\\ 
        3      &  0    &  0    &  598  &  180   &  0    &  0    &  0.971  & 0.971\\ 
        4      &  0    &  0    &  0    & 1525  &  0    &  0    &  1.000  & 0.988\\ 
        5      &  0    &  0    &  0    &  0    &  671  &  30    &  0.996  & 0.996\\ 
        6      &  0    &  180   &  0    &  0    &  0    &  781  &  0.977  & 0.974\\ \hline
    \end{tabular}
    \caption{Pixel-wise Confusion Matrix (Vertical: Actual, Horizontal: Predicted)}
\end{table}

Salinas-B is a simpler segmentation case, prior knowledge about the scene confirms the aim is to segment the image into $n_e = 5$ segments corresponding to the following vegetation classifications in the image.
\begin{table}[H]
    % \caption{Groundtruth classes for the Salinas-A scene and their respective samples number}
    \centering
    \label{tab:salinas_b_classes}
    \begin{tabular}{|c|c|c|}
    \hline
    \textbf{Label} & \textbf{Class} & \textbf{Samples} \\
    \hline
    1 & Fallow Rough Plow & 490 \\
    2 & Fallow Smooth & 989 \\
    3 & Stubble & 1627 \\
    4 & Celery & 1127 \\
    5 & Vineyard Untrained  & 2891 \\
    \hline
    \end{tabular}
    \caption{Ground truth classes for the Salinas-B scene and their respective samples number}
\end{table}
\begin{figure}[H]
    \centering
    \includegraphics[width=10cm]{salinas-b-gt.png}  % Adjust width and filename
    \caption{Salinas-B Ground Truth Labels}
    \label{salina-b}  % Optional label for referencing
\end{figure}


Using an initial selection of $n_s = 330$ superpixels with shape parameter $m=2$ and unmixing parameters $\mu = 1$ and $\beta = 0.0025$, a grid search was applied to $\sigma$ and $\kappa$ such that $\sigma \in [0.1, 0.001]$ and $\kappa \in [15, 40]$, to which $\sigma = 0.0025$ and $\kappa = 10$ produced the most optimal results with an overall accuracy of $0.998$. 
\begin{figure}[H]
    \centering
    \includegraphics[width=10cm]{salinas-b-labelled.png}  % Adjust width and filename
    \caption{Salinas-B Algorithm Results}
    \label{salina-b-results}  % Optional label for referencing
\end{figure}
\begin{table}[H]
    \centering
    \label{tab:salinas_b_cfm}
    \begin{tabular}{|c|ccccc|c|c|}
        \hline
         & \textbf{1} & \textbf{2} & \textbf{3} & \textbf{4} & \textbf{5} & \textbf{Accuracy} & \textbf{IoU} \\ \hline
        1     &  490    &  0    &  0    &  0        &  0    &  1.00 & 0.992  \\ 
        2      &  0    & 989  &  0    &  0        &  0    &  1.000   & 1.00\\ 
        3      &  0    &  0    &  1621  &  6       &  0    &  0.996  & 0.996\\ 
        4      &  0    &  0    &  0    & 1127    &  0    &  1.000  & 0.995\\ 
        5      &  4   &  0    &  0    &  0      &  2887    &  0.999  & 0.999\\ \hline
    \end{tabular}
    \caption{Pixel-wise Confusion Matrix (Vertical: Actual, Horizontal: Predicted)}
\end{table}
Again, the algorithm itself performs well across all classes, with high individual segment accuracy scores as well as IoU scores, demonstrating it's ability to perform segmentation in scenarios where similar spectral features are shared among materials and materials span across the image.

    



\subsubsection{Qualitative Evaluation on Samson}
Our proposed segmentation algorithm was applied to the Samson A,B and C datasets. While this dataset does not have labelled data provided with it, a qualitative comparison can be performed, with particular focus on noting the ability of the algorithm to distinguish highly mixed pixels along the shorelines of the image and the ability to capture spatial complex segments.

For the Samson datasets, prior knowledge about the scene confirms the aim is to segment the image into $n_e = 3$ segments corresponding to the Water, Dirt and Tree classifications. Using an initial selection of $n_s = 961$ superpixels with shape parameter $m=3$ and unmixing parameters $\mu = 1$ and $\beta = 0.005$, a grid search was applied to $\sigma$ and $\kappa$ such that $\sigma \in [0.1, 0.001]$ and $\kappa \in [15, 40]$, to which $\sigma = 0.015$ and $\kappa = 30$ produced the most optimal results in terms of visual coherence with respect to the original image. 
\begin{figure}[H]
    \centering
    \includegraphics[width=15cm]{samsonabc-results.png}  % Adjust width and filename
    \caption{Algorithm Results on Samson (Red is Water, Purple is Trees, Yellow is Dirt)}
    \label{samson-abc-results}  % Optional label for referencing
  \end{figure}
It can be seen that the algorithm performs well and creates segmentations analagous to the distribution of materials within the scene, with the additional benefit of being able to form noncontigous structures when the spatial limit $\kappa$ is set to be higher, allowing more refined segmentation when pockets of materials exist within eachother. 




% n_superpixels = 1000 #2500
% slic_m_param = 3  #2
% sigma_param = 0.015 # 0.1 -> 0.001           #0.01
% spatial_limit = 30# 15 -> 25 in steps of 5 #15
% spatial_beta_param = 0.0025
% spatial_dmax_param = spatial_limit
% ne = 3#number of endmembers


% \subsubsection{Qualitative Evaluation on Biomedical Autofluorescence Data}

% \clearpage
% \subsection{Algorithm Comparison}

\clearpage
% #############################################
% 
% 
% 
% 
% 
% #############################################

\clearpage
% #############################################
% 
% 
% 
% 
% 
% #############################################
\section{Conclusions}
The proposed algorithm works off the key assumption for future biomedical hyperspectral images that the eye tissues of interest lie in spatially coherent regions and that pixels near each other share similar spectral characteristics. At the heart, a segmentation approach aims to minimize inter segment variance and maximize intrasegment variance. In the context of hyperspectral imaging, the goal of creating accurate segmentations and maximizing accuracy are often the same. As to efficiency directly, the pre clustering step is crucial to allowing an efficient implementation of the graph approach to segmentation. The relaxed normalized cuts approach we implement operates in near cubic time complexity, by selecting a certain average pixel to superpixel ratio, the running time of the algorithm is reduced cubically with respect to a graph approach without superpixel generation.

The experimental results on real-world satellite hyperspectral datasets demonstrate the effectiveness of the proposed approach. It achieves a high degree of accuracy in material identification while having a high degree of flexibility in terms of choices on parameters to produce the final segmentation output. This method offers a valuable tool for researchers and practitioners working in diverse fields that utilize hyperspectral imaging technology in unlabelled scenarios, such as remote sensing and biomedical hyperspectral imaging. Future research directions would be to explore different constructions of the spatial-spectral affinity matrix presented in \eqref{nc:spatial-spectral-mtx}. While the spectral angle metric is useful in many hyperspectral scenarios, the argument can be made that spectral angle alone cannot capture the intricate differences between normalized spectra for certain fields of study, and instead, a mixed of different measures must be considered.

In conclusion, this work presented a novel graph-based segmentation approach that effectively addresses the challenges associated with blind segmentation in hyperspectral images. The proposed method leverages the strengths of both spectral similarity and spatial proximity to achieve accurate segmentation while maintaining computational efficiency. By incorporating the estimated abundance information as a supportive feature, the algorithm refines the segmentation process, leading to superior results in both accuracy and efficacy.


\end{document}