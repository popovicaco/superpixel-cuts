\documentclass[12pt]{article}
\usepackage{graphicx} % Required for inserting images
\usepackage{amsmath}
\usepackage{amssymb}
\usepackage[margin=1in]{geometry}
\usepackage{algorithm2e}
\usepackage{fancyhdr}

%############################ SETTINGS
\RestyleAlgo{ruled}
% \pagestyle{fancy} % Apply headers to all pages
% \fancyhf{} % Clear all header and footer definitions
% % \fancyhead[L]{\small Adaptive Superpixel Cuts for Hyperspectral Images}
% % \fancyhead[R]{\small \thepage}
\DeclareMathOperator*{\argmax}{arg\,max}
\DeclareMathOperator*{\argmin}{arg\,min}
% #############################################
% 
% 
% 
% 
% 
% #############################################
\title{Adaptive Superpixel Cuts for Hyperspectral Images}
\author{Aleksandar Popovic}
\date{February 2024}

\begin{document}

\maketitle

\begin{abstract}
    Blind segmentation in hyperspectral images is a challenging problem. Many traditional methods suffer from poor identification of materials and expensive computational costs, which can be partially eased by trading the accuracy with efficiency.

    In this paper, we propose a novel graph-based algorithm for segmentation in hyperspectral images. Utilizing the fact that pixels in a local region are likely to have similar spectral features, a pre-clustering algorithm is used to extract the homogenous regions, called superpixels. After extracting the superpixels, a weighted graph is constructed with the weights representing both the spectral similarity and spatial distance between each superpixel and its neighbors. The normalized graph cuts algorithm is then used to perform an initial segmentation of the image. To effectively extract the material information in the superpixels, the mean spectra in each segment is used to estimate the abundance of each endmember in each superpixel using a graph regularized hyperspectral unmixing algorithm. The resulting abundance information is used as a supportive feature, which when combined with the spectral features, form a new spectral feature vector for each superpixel. Using this new feature vector, the weighted graph is once again constructed and the normalized cuts algorithm is applied, resulting in a final segmentation of the image.

    Experiments on a real hyperspectral datasets illustrate great potential of the proposed method in terms of accuracy and efficiency.
\end{abstract}

\clearpage
% #############################################
% 
% 
% 
% 
% 
% #############################################
\tableofcontents

\clearpage
% #############################################
% 
% 
% 
% 
% 
% #############################################
\section{Introduction}

\subsection{Motivations}

\clearpage
% #############################################
% 
% 
% 
% 
% 
% #############################################
\section{Background} \label{Background}
\subsection{Hyperspectral Imaging: Basics} \label{Basics}
Hyperspectral Imaging is an advanced technology that combines the power of spectroscopy and digital imaging to capture and analyze detailed information about objects of interest over a large number of spectral wavelengths beyond the visible spectrum. 
\subsubsection{The Hyperspectral Cube} \label{Cube}
In traditional, RGB based imaging systems, an image can be represented by a 3-dimensional tensor of shape $(n_x, n_y, 3)$, where the last dimension corresponds to the color channel the image was captured in. From a mathematical point of view, a hyperspectral image, denoted by $\mathbf{X}$, is a tensor of shape $(n_x, n_y, n_\lambda)$ with nonnegative entries. Each pixel in the tensor is represented using a vector $\mathbf{x} \in \mathbb{R}_+^{n_\lambda}$. From a physical point of view, the first two dimensions in $\mathbf{X}$ represent the spatial coordinates  of the pixels, while the last dimension represents the specific wavelength band the spectral intensity, reflectance or transmittance measurements were taken at.

The tensor $\mathbf{X}$ can be transformed into a two-dimensional matrix of shape $(n_p, n_\lambda)$, denoted $\mathbf{X}_f$, where $n_p = n_x n_y$ is the total number of pixels in the image, with each column representing the pixel at index $(i,j)$ in $\mathbf{X}$, arranged across the first spectral dimension, then the second. Formally, $
\mathbf{X}_f = \left[ \mathbf{x}_{(0,0)},  \cdots, \mathbf{x}_{(n_x,0)}, \cdots, \mathbf{x}_{(0,n_y)}, \cdots, \mathbf{x}_{(n_x, n_y)} \right]. 
$



\clearpage
\subsection{Superpixel Generation} \label{Superpixel}
As mentioned in Section (2.1), high spectral resolution is a common constraint in hyperspectral image analysis, with time complexity of the algorithms commonly scaling polynomially with the number of pixels in the image. 

Before the introduction of neural network based solutions in computer vision applications, there was interest in preclustering images into locally homogeneous regions called superpixels [superpixel ref]. Addressing the main motivation of reducing the overall granularity of the data, superpixels are also shown to preserve spectral information, adhere to spatial features in the image, and introduce robustness against noise in subsequent analysis tasks.


\subsubsection{Simple Linear Iterative Clustering} \label{SLIC}
In this section, we will introduce the Simple Linear Iterative Clustering (SLIC) algorithm. The algorithm is a special case of the k-means algorithm adapted to the task generating superpixels in a $5$-dimensional space, where the first $3$ dimensions correspond to the the pixel color vector in the CIELAB colorpsace, and last $2$ dimensions correspond to the spatial coordinates $(i,j)$ of the pixel in the image.. Formally, we restructure each pixel $\mathbf{x}_{(i,j)} = [\mathbf{x}_l, \mathbf{x}_a, \mathbf{x}_b]$ into the form $\tilde{\mathbf{x}}_{(i,j)} = [\mathbf{x}_l, \mathbf{x}_a, \mathbf{x}_b, i, j]$. With this modified feature vector $\tilde{\mathbf{x}}$, we incorporate both spectral and spatial information into the clustering, however, while the spectral information has bounds on it's values, the spatial information depends on the size of the image. 

Taking as an input the desired number of superpixels $n_s$, for an image with $n_p = n_x n_y$ pixels, each superpixel would be composed of approximately $n_s / n_p$ pixels. Assuming the superpixels lie on a grid, a superpixel centroid would occur at every grid interval $S = \sqrt{n_s/n_p}$. At the onset of the algorithm, a grid of $n_s$ superpixel centers $\mathbf{C}_n = [\mathbf{c}_{l}, \mathbf{c}_{a}, \mathbf{c}_{b}, i, j]$ where $n = 1, \cdots, n_s$ are sampled across the image with regular grid intervals $S$. To avoid sampling noisy pixels, clusters are moved to the lowest gradient position in a $3 \times 3$ neighborhood where the image gradient is calcuated, using the original spectral vector $x$ in the CIELAB color space as:
\begin{equation}
    \label{eq:slic-gradient}
    \mathbb{G}(i,j) = \|\mathbf{x}_{(i+1,j)} - \mathbf{x}_{(i-1,j)} \|^2 + \|\mathbf{x}_{(i,j+1)} - \mathbf{x}_{(i,j-1)} \|^2
\end{equation}

After initialization, a modified distance measure is proposed to enforce color similarity and spatial extent within the superpixels. Since the approximate area of each superpixel is $S^2$, it is assumed that pixels associated with a superpixel lie within a $2S \times 2S$ neighborhood of the superpixel centroid. Introducing the parameter $m$ to control the compactness and shape of the superpixels, the modified distance is then calculated as 
\begin{equation}
    \label{eq:slic-cielab-distance}
    \mathbb{D}(x, y) = \|\mathbf{x}_{lab} - \mathbf{y}_{lab}\|^2 + \frac{m}{S}\|\mathbf{x}_{ij} - \mathbf{y}_{ij}\|^2
\end{equation}

Each pixel is the image is associated with the nearest cluster whose search area overlaps this pixel. After all pixels are associated with a cluster, a new center is computed as the average feature vector of all the pixels belonging to the cluster. This is repeated for a set number of iterations. After exhausting all iterations, a final step is performed by relabelling disjoint segments with the labels of the largest neighboring cluster. This step is optional as disjoint segments tend to not occur for larger inputs of $n_s$ and $m$.

\begin{algorithm}
    \caption{SLIC Superpixel Algorithm}
    
    \textbf{Input}: $\mathbf{X}_f$, $m > 0$, $n_s > 0$, $n_{\text{iters}} > 0$

    \textbf{Initialize:} $\mathbf{C}_n = [\mathbf{c}_{l}, \mathbf{c}_{a}, \mathbf{c}_{b}, i, j]$ where $n = 1, \cdots, n_s$ by sampling pixels at regular grid intervals $S$. Perturb cluster centers to lowest gradient position in a $3 \times 3$ neighborhood according to \eqref{eq:slic-gradient} \\
    
    \For{$k = 1$ \KwTo $n_{\text{iters}}$}{ 
        Assign best matching pixels from a $2S \times 2S$ neighborhood around clusters $C_k$ according to distance measure \eqref{eq:slic-cielab-distance}. Compute new cluster centers according to average of all pixels belonging to cluster.
    }
    \textbf{Optional:} Relabel disjoint segments.
\end{algorithm}

The SLIC algorithm is shown to produce meaningful and noise-robust segments in traditional computer vision applications. This algorithm proves useful in Section \ref{Algorithm Superpixels} when adapted as a pre-clustering step in the hyperspectral domain.


\clearpage
\subsection{Hyperspectral Unmixing} \label{Unmixing Intro}
In hyperspectral imaging applications, there is emphasis on estimating the relative abundance of a given representantive materical, called an endmember, within each pixel. 

Unmixing results often give more detailed information about the overall composition of a hyperspectral scene with respect to simple segmentation. This section will introduce the foundational framework behind unmixing and abundance estimation.
\subsubsection{Linear Mixing Model}\label{LMM}
In reality, most pixels in a hyperspectral image capture a mixture of spectra reflected from various materials present within the spatial area, due to constraints with how large a spatial resolution can be acheived. 

\begin{figure}[h]
    \caption{Image of LMM goes here!}
    \label{fig:figure2}
\end{figure}


The foundational model behind hyperspectral unmixing is the linear mixing model, which dictates that spectra of every pixel $\mathbf{x} \in \mathbb{R}_+^{n_b}$ in a hyperspectral image is a linear combination of a set of $n_e$ spectra, $\mathbf{m} _1, \mathbf{m} _2, \cdots, \mathbf{m} _{n_e} \in \mathbb{R}_+^{n_b}$, from pure representative materials, called endmembers, with weights $a_1, a_2, \cdots, a_{n_e} \in \mathbb{R}$. Denoting $\mathbf{M} = [\mathbf{m} _1 | \mathbf{m} _2 | \cdots |\mathbf{m} _{n_e}] \in \mathbb{R}_+^{n_b \times n_e}$ and $\mathbf{a} = [a_1, a_2, \cdots ,a_{n_e}]^T \in \mathbb{R}^{n_e}$, the linear mixing model is formulated as follows:

\begin{equation}
    \label{lmm:model}
    \mathbf{x} = \mathbf{M} \mathbf{a} + \mathbf{\epsilon}.
\end{equation}

While this model serves useful, there is no direct physical interpretation to the weights in $\mathbf{a}$, instead, we aim to estimate the physical proportion, called the abundance, of each endmember within each pixel by imposing two constraints on the entries in $\mathbf{a}$. The abundance nonnegativity constraint (ANC) requires that the entries in $\mathbf{a}$ must be greater than or equal to zero, while the abundance sum-to-one constraint (ASC) requires that the entries in $\mathbf{a}$ sum to $1$. Combining the two constraints, we have an extension of the linear mixing model

\begin{equation}
    \label{lmm:abund-lmm}
    \mathbf{x} = \mathbf{M} \mathbf{a} + \mathbf{\epsilon} \quad \text{ s.t } \mathbf{a} \in \mathbb{R}_+^{n_e} \text{ and } \|\mathbf{a}\|_1 = 1.
\end{equation}

The linear mixing model can be additionally be extended from a per pixel basis onto a collection of $n_p$ pixels $\mathbf{X} = [\mathbf{x}_1 | \mathbf{x}_2 | \cdots | \mathbf{x}_{n_p}] \in \mathbb{R}_ +^{n_b \times n_p}$, with each pixel $\mathbf{x}_i$ having a corresponding abundance vector $\mathbf{a}_i$. Arranging the abundance vectors into an abudance matrix $\mathbf{A} = [\mathbf{a}_1 | \mathbf{a}_2 | \cdots | \mathbf{a}_{n_p}] \in \mathbb{R}^{n_e \times n_p}$, we denote the ANC-ASC constraint using the set $ \Delta = \{ \mathbf{A} \in \mathbb{R}_+^{n_e \times n_p} \mid \mathbf{1}_{n_e}^T \mathbf{A} = \mathbf{1}_{n_p}\} $. This new extension of the linear mixing model to a collection of pixels that will be used for the following sections

\begin{equation}
    \label{lmm:abund-lmm-collection}
    \mathbf{X} = \mathbf{M} \mathbf{A} + \epsilon \quad \text{ s.t } \mathbf{A} \in \Delta
\end{equation}

The linear mixing model is an objectively simple model, which enforces a linear relationship between the spatial mixing of endmembers through assuming that pixels lie on a flat plane. This is almost never the case, however the model remains a efficient and powerful tool for extracting spectral information from a scene. 
\subsubsection{Abundance Estimation}\label{AE}
Given the linear mixing model as formulated in Section \ref{LMM}, in traditional hyperspectral imaging tasks, both $\mathbf{M}$ and $\mathbf{A}$ are unknown. Often, researchers aim to estimate $\mathbf{M}$ first, as spectral signatures collected from endmembers in same scene under the same conditions will be almost identical.

Notably, in the field of remote sensing, effort has been made to create a library of spectral signatures derived vegetation and minerals from land cover images, allowing focus to be made solely in estimating $\mathbf{A}$. This section will cover the scenario where $\mathbf{M}$ is known and $\mathbf{A}$ is to be estimated. The task is referred to as abundance estimation and continues to be an active area of research.



As mentioned in the beginning of \ref{Unmixing Intro}, 
\subsubsection{Alternating Direction Method of Multipliers}\label{ADMM Intro}
Alternating Direction Method of Multipliers, or ADMM, introduced by Boyd et al. [REF] is a framework for solving convex optimization problems of the form:
\begin{equation}
    \label{admm:prob}
    \begin{aligned}
    &\text{minimize} \quad f(x) + g(z) \\
    &\text{subject to} \quad Ax + Bz = c
    \end{aligned}
\end{equation}
with variables $x \in \mathbb{R}^{n}$ and $z \in \mathbb{R}^{m}$, where $A \in \mathbb{R}^{p \times n}$, $B \in \mathbb{R}^{p \times m}$ and $c \in \mathbb{R}^{p}$. $f$ and $g$ are assumed to be convex. The aim of ADMM is to incorporate the decomposability of the dual ascent method into the superior convergence properties of method of multipliers. To allow for this, ADMM introduces the corresponding augmented Lagrangian $\mathcal{L}_{\mu}$ defined as:
\begin{equation}
    \label{admm:lagrangian}
    \mathcal{L}_{\mu}(x,z,y) = f(x) + g(z) + y^T(Ax + Bz - c) + \frac{\mu}{2} \|Ax + Bz - c\|_2^2
\end{equation}
where $\mu > 0$ is augmented lagrangian convergence parameter and $y \in \mathbb{R}^{p}$ is the corresponding dual variable. Scaling with $u  = \frac{1}{\mu} y $ gives the following equivalent definition: 
\begin{equation}
    \label{admm:lagrangian-real}
    \mathcal{L}_{\mu}(x,z,u) = f(x) + g(z) + \frac{\mu}{2} \|Ax + Bz - c + u\|_2^2.
\end{equation}

ADMM aims to minimize scaled form of $\mathcal{L}_{\mu}$ by alternating minimizations with respect to $x$, $y$, and $u$ by performing the following updates:

\begin{equation}
    \label{admm:updates}
    \begin{aligned}
        x^{(k+1)} &= \argmin_{x} \mathcal{L}_{\mu}(x,z^{(k)},u^{(k)})  \\
        z^{(k+1)} &= \argmin_{z} \mathcal{L}_{\mu}(x^{(k+1)},z,u^{(k)}) \\
        u^{(k+1)} &= u^{(k)} + Ax^{(k+1)} + Bz^{(k+1)} - c.
    \end{aligned} 
\end{equation}

Under mild conditions on $f$ and $g$, ADMM can be shown to provide guarenteed objective and residual convergence, independent on choice of $\mu$. For lax choices of $\mu$, the algorithm provides modest accuracy solutions in a relatively low number of iterations, favorable to tasks in statistical learning where parameter estimation often yields little improvement to results. The algorithm allows practioners to put focus on efficient implementations to the minimization problems for $x$ and $z$, which proves useful in the following section.


\subsubsection{ADMM Based Abundance Estimation}\label{Block ADMM}
In Section \ref{AE}, the abundance estimation problem for a collection of pixels $\mathbf{X}$ given the endmember spectra matrix $\mathbf{M}$ was known was stated as follows:

\begin{equation*}
    A = \argmin_{A \in \mathbb{R}^{n_e \times n_p}} \frac{1}{2}\|\mathbf{MA} - \mathbf{X}\|_F^2 + \chi_\Delta(\mathbf{A}) + J(\mathbf{A}).
\end{equation*}

The goal of this section is to demonstrate how this problem can be equivalently represented in a form where ADMM can directly be applied. The section itself operates under the assumption that $J$ is a convex function, for nonconvex choices of $J$, convergence is not guarenteed. The approach to transforming \eqref{ae:ae-min-1} is to first introduce matrices $\mathbf{U} \in \mathbb{R}^{n_e \times n_p}$, $\mathbf{V}_1 \in \mathbb{R}^{n_b \times n_p}$, $\mathbf{V}_2 \in \mathbb{R}^{n_e \times n_p}$, and $\mathbf{V}_{3} \in \mathbb{R}^{n_e \times n_p}$ and rewrite the problem into the equivalent form 
\begin{equation}
    \label{ae:equivalent-admm-1}
    \begin{aligned}
        \underset{\mathbf{U}, \mathbf{V}_1, \mathbf{V}_2, \mathbf{V}_2}{\textbf{minimize }} & \frac{1}{2} \|\mathbf{V}_1 - X \|_F^2 + \chi_{\Delta}(\mathbf{V}_2) + J(\mathbf{V}_3) 
        \\         
        \textbf{subject to } & \mathbf{V}_1 = \mathbf{MU} \\
        &\mathbf{V}_2 = \mathbf{U} \\
        &\mathbf{V}_{3} = \mathbf{U}.
   \end{aligned}
\end{equation}
Further rewriting the problem, the application of ADMM can clearly be seen:
\begin{equation}
    \label{ae:equivalent-admm-2}
    \begin{aligned}
        \underset{\mathbf{U},\mathbf{V}}{\textbf{minimize }} & g(\mathbf{V})
        \\         
        \textbf{subject to } & \mathbf{GU} + \mathbf{BV} = \mathbf{0}
   \end{aligned}
\end{equation}
where 
$
g(\mathbf{V}) = \frac{1}{2} \|\mathbf{V}_1 - X \|_F^2 + \chi_{\Delta}(\mathbf{V}_2) + J(\mathbf{V}_3) 
$, 
$$
\mathbf{V} = \begin{bmatrix}
\mathbf{V}_1 &  &   \\
 &\mathbf{V}_2&   \\
  &  & \mathbf{V}_3 \\
\end{bmatrix},
\quad
\mathbf{G} = 
\begin{bmatrix}
\mathbf{M}\\ 
\mathbf{I}\\ 
\mathbf{I}\\ 
\end{bmatrix},
\quad
\mathbf{B} = 
\begin{bmatrix}
-\mathbf{I} &  &  \\
  &-\mathbf{I}&  \\
&  & -\mathbf{I} \\
\end{bmatrix}.
$$

With parameter $\mu > 0$, the augmented lagrangian $\mathcal{L}_\mu$ for the abundance estimation problem 

\clearpage
\subsection{Spectral Clustering} \label{Spectral Clustering}
\subsubsection{Normalized Cuts} \label{Normalized Cuts}

\clearpage
% #############################################
% 
% 
% 
% 
% 
% #############################################
\section{Adaptive Superpixel Cuts} \label{Algorithm Intro}

\subsection{Dataset Preprocessing} \label{Algorithm Preprocessing}
\subsubsection{Singular Value Decomposition} \label{SVD}
\subsubsection{Layer Normalization} \label{Normalization}

\clearpage
\subsection{Algorithm Overview} \label{Algorithm Overview}
\subsubsection{Superpixel Generation} \label{Algorithm Superpixels}
\subsubsection{Construction of the Affinity Matrix $\mathbf{W}$}\label{Algorithm Laplacian}
\subsubsection{Normalized Cuts Algorithm}\label{Algorithm Ncuts}
\subsubsection{Graph Regularized Fully Constrained Unmixing}\label{Algorithm Unmixing}
\subsubsection{Feature Vector Creation}\label{Algorithm FV}
\subsubsection{Parameter Selection}\label{Parameters}

\clearpage
% #############################################
% 
% 
% 
% 
% 
% #############################################
\section{Experimental Results}

\subsection{Implementation Details}

\clearpage
\subsection{Geospatial Testing Datasets}
\subsubsection{Samson Datasets}
\subsubsection{Salinas Datasets}

\clearpage
\subsection{Algorithm Evaluation}
\subsubsection{Qualitative Evaluation on Samson}
\subsubsection{Quantitative Evaluation on Salinas}

\clearpage
\subsection{Algorithm Comparison}

\clearpage
% #############################################
% 
% 
% 
% 
% 
% #############################################
\section{Applications in Biomedical Imaging}

\subsection{Dr Yucel Motivations}
\subsection{Results on Biomedical Hyperspectral Images}

\clearpage
% #############################################
% 
% 
% 
% 
% 
% #############################################
\section{Conclusions}

\end{document}