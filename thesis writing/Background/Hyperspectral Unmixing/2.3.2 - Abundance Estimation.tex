Given the linear mixing model as formulated in Section \ref{LMM}, in traditional hyperspectral imaging tasks, both $\mathbf{M}$ and $\mathbf{A}$ are unknown. Often, researchers aim to estimate $\mathbf{M}$ first, as spectral signatures collected from endmembers in same scene under the same conditions will be almost identical.

Notably, in the field of remote sensing, effort has been made to create a library of spectral signatures derived vegetation and minerals from land cover images, allowing focus to be made solely in estimating $\mathbf{A}$. This section will cover the scenario where $\mathbf{M}$ is known and $\mathbf{A}$ is to be estimated. The task is referred to as abundance estimation and continues to be an active area of research.



As mentioned in the beginning of \ref{Unmixing Intro}, 