In this section, we will introduce the Simple Linear Iterative Clustering (SLIC) algorithm. The algorithm is a special case of the k-means algorithm adapted to the task generating superpixels in a $5$-dimensional space, where the first $3$ dimensions correspond to the the pixel color vector in the CIELAB colorpsace. Formally, each pixel $\mathbf{x}_{(i,j)} = [x_l, x_a, x_b]$. 

Taking as an input the desired number of superpixels $n_s$, for an image with $n_p = n_x n_y$ pixels, each superpixel would be composed of approximately $n_s / n_p$ pixels. Assuming the superpixels lie on a grid, a superpixel centroid would occur at every grid interval $S = \sqrt{n_s/n_p}$. At the onset of the algorithm, a grid of $n_s$ superpixel centers $\mathbf{C}_n = [c_{l}, c_{a}, c_{b}]$ where $n = 1, \cdots, n_s$ are sampled across the image with regular grid intervals $S$. To avoid sampling noisy pixels, clusters are moved to the lowest gradient position in a $3 \times 3$ neighborhood where the image gradient is calcuated, using the original spectral vector $\mathbf{x}$ in the CIELAB color space as:
\begin{equation}
    \label{eq:slic-gradient}
    \mathbb{G}(i,j) = \|\mathbf{x}_{(i+1,j)} - \mathbf{x}_{(i-1,j)} \|^2 + \|\mathbf{x}_{(i,j+1)} - \mathbf{x}_{(i,j-1)} \|^2
\end{equation}

After initialization, a modified distance measure is proposed to enforce color similarity and spatial extent within the superpixels. Since the approximate area of each superpixel is $S^2$, it is assumed that pixels associated with a superpixel lie within a $2S \times 2S$ neighborhood of the superpixel centroid. Introducing the parameter $m$ to control the compactness and shape of the superpixels, the modified distance is then calculated as 
\begin{equation}
    \label{eq:slic-cielab-distance}
    \mathbb{D}(\mathbf{x}, \mathbf{c}_n) = \|\mathbf{x} - \mathbf{c}_n\|_2^2 + \frac{m}{S}d_{\text{spatial}}(\mathbf{x}, \mathbf{c}_n)^2
\end{equation}

Each pixel is the image is associated with the nearest cluster whose search area overlaps this pixel. After all pixels are associated with a cluster, a new center is computed as the average feature vector of all the pixels belonging to the cluster. This is repeated for a set number of iterations $k_\text{max}$. 

\begin{algorithm}[H]
    \caption{SLIC Superpixel Algorithm}
    \textbf{Input}: \\
    \quad CIELAB Image $\mathbf{X}$
    \quad  $m > 0$, $n_s > 0$, $k_{\text{max}} > 0$
    \\
    \textbf{Initialize:} $\mathbf{c}_n = [c_{l}, c_{a}, c_{b}, i, j]$ where $n = 1, \cdots, n_s$ by sampling pixels at regular grid intervals $S$. Perturb cluster centers to lowest gradient position in a $3 \times 3$ neighborhood according to \eqref{eq:slic-gradient} \\
    
    \For{$k = 1$ \KwTo $k_{\text{max}}$}{ 
        Assign best matching pixels from a $2S \times 2S$ neighborhood around clusters $C_k$ according to distance measure \eqref{eq:slic-cielab-distance}. \\
        Compute new cluster centers according to average vector of all pixels belonging to cluster.
    }
    \textbf{Output}: Superpixeled Image $\mathbf{X}_s = [\mathbf{c}_1, \mathbf{c}_2, \dots, \mathbf{c}_{n_s}]$
\end{algorithm}

The SLIC algorithm is shown to produce meaningful and noise-robust segments in traditional computer vision applications. This algorithm proves useful in Section \ref{Algorithm Superpixels} when adapted as a spatial preprocessing step in the hyperspectral domain.
