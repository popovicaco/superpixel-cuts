In traditional, RGB based imaging systems, an image can be represented by a 3-dimensional tensor of shape $(n_x, n_y, 3)$, where the last dimension corresponds to the color channel the image was captured in. From a mathematical point of view, a hyperspectral image, denoted by $\mathbf{X}$, is a tensor of shape $(n_x, n_y, n_\lambda)$ with nonnegative entries. Each pixel in the tensor is represented using a vector $\mathbf{x} \in \mathbb{R}_+^{n_\lambda}$. From a physical point of view, the first two dimensions in $\mathbf{X}$ represent the spatial coordinates  of the pixels, while the last dimension represents the specific wavelength band the spectral intensity, reflectance or transmittance measurements were taken at.
