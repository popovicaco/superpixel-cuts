In hyperspectral imaging, near contiguous narrow band spectral information is measured for each spatial pixel of an image collected over a scene. Spectral information can be quantified by multiple measures. Traditionally, spectral radiance, being the energy emitted or reflected by a surface over a large number of spectral wavelengths, has been the measure of choice for physical applications. Utilizing this spectral information, physical and chemical properies of materials can be deduced and insights can be made about the overall composition of the images, resulting in applications across various fields.

Analysis in hyperspectral images has traditionally been a computationally expensive and difficult task due to algorithms scaling in both the spatial and spectral resolution of the images. This section will focus on building a relevant background for common preclustering, abundance estimation, and segmentation techniques that are common in hyperspectral image analysis.

