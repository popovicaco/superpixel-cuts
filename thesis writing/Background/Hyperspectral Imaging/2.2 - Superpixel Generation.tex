As mentioned in Section (2.1), high spectral resolution is a common constraint in hyperspectral image analysis, with time complexity of the algorithms commonly scaling polynomially with the number of pixels in the image. 

Before the introduction of neural network based solutions in computer vision applications, there was interest in preclustering images into locally homogeneous regions called superpixels [superpixel ref]. Addressing the main motivation of reducing the overall granularity of the data, superpixels are also shown to preserve spectral information, adhere to spatial features in the image, and introduce robustness against noise in subsequent analysis tasks.

In non-hyperspectral applications, the superpixel algorithm uses spectral information in the 3-dimensional CIELAB colorspace and spatial information in the form of pixel coordinates.

